\documentclass[12pt,a4paper,bibliography=totocnumbered,listof=totocnumbered]{article}
% u.U. muss Koma-Skript Package ueber MikTeX deinstalliert und neu installiert werden
% Hilft das nicht, so sollte statt scrartcl die Dokumentenklasse article verwendet werden
\input{lib/includes}
% Kopf- und Fusszeile
\renewcommand{\sectionmark}[1]{\markright{#1}}
\renewcommand{\leftmark}{\rightmark}
\pagestyle{fancy}
\lhead{}
\chead{}
\rhead{\thesection\space\contentsname}
\lfoot{}
\cfoot{}
\rfoot{\ \linebreak Seite \thepage}
\renewcommand{\headrulewidth}{0.4pt}
\renewcommand{\footrulewidth}{0.4pt}

% Vorspann
\renewcommand{\thesection}{\Roman{section}}
\renewcommand{\theHsection}{\Roman{section}}
\pagenumbering{Roman}

\newcommand{\folgen}[1]{
\ensuremath
#1
}

\newcommandx{\student}[3][]{
	\def\studentName{#1}%
	\def\studentMatnr{#2}%
	\def\studentStudiengang{#3}%
}


\newcommandx{\MyTitelseite}[8][]{
\thispagestyle{empty}
\includegraphics[scale=0.2]{pics/oth-logo.png}\hfill
\IfFileExists{#1}{\includegraphics[scale=0.5]{#1}}{}
\begin{center}
\ifthenelse{\equal{#2}{2}}{ % then
	\vspace*{2cm}
	\Large
	\textbf{Ostbayerische Technische Hochschule Regensburg}\\
	\textbf{Fakultät für Informatik und Mathematik}\\
	\vspace*{2cm}
	\Huge
	\textbf{#3}\\[1em]
	\large
	Zur Erlangung des akademischen Grades des\\
	\ifthenelse{\equal{#3}{Bachelorarbeit}}{Bachelor of Science (B.Sc.)}{Master of Science (M.Sc.)}\\
	\vspace*{1cm}
	\Large
	\textbf{#4}\\
}{ % else
	\vspace*{1cm}
	\Large
	\textbf{#4}\\
	\vspace*{2cm}
	\large
	An der Fakultät für Informatik und Mathematik der\\
	Ostbayerischen Technischen Hochschule Regensburg\\
	im Studiengang\\
	\studentStudiengang\\[2em]
	eingereichte\\
	\vspace*{1cm}
	\Large
	\textbf{#3}\\[2em]
	\large
	zur Erlangung des akademischen Grades des\\
	\ifthenelse{\equal{#3}{Bachelorarbeit}}{Bachelor of Science (B.Sc.)}{Master of Science (M.Sc.)}
	\vspace*{1cm}
	\Large
}
	\vfill
	\normalsize
	%\newcolumntype{x}[1]{>{\raggedleft\arraybackslash\hspace{0pt}}p{#1}}
	\begin{tabular}{rl}%{6cm}p{7.5cm}}
	    \rule{0mm}{1ex}\textbf{Vorgelegt von:} & \studentName \\
		\rule{0mm}{1ex}\textbf{Matrikelnummer:} & \hspace*{-0.5em}\begin{tabular}[t]{r}\studentMatnr\end{tabular} \\ 
		\ifthenelse{\equal{#2}{1}}{~\\}{\rule{0mm}{1ex}\textbf{Studiengang:} & \studentStudiengang \\[2em]}
		\rule{0mm}{1ex}\textbf{Erstgutachter:} & #5 \\ 
		\rule{0mm}{1ex}\textbf{Zweitgutachter:} & #6 \\[2em]
		\rule{0mm}{1ex}\textbf{Abgabedatum:} & #7 \\ 
	\end{tabular} 
\end{center}
\pagebreak
}

\newsavebox\mybox
\savebox\mybox{%
  \tikz{
    \draw[ultra thick,red] (-4pt,-4pt) -- (4pt,4pt);
    \draw[ultra thick,red] (-4pt,4pt) -- (4pt,-4pt);
  }%
}  

\newcommand{\TopAlign}[1]{\adjustbox{valign=t}{#1}}
\newcolumntype{T}{>{\collectcell{\TopAlign}}c<{\endcollectcell}}



\forestset{
  upsideTria/.style={
    node format={
      \noexpand\node [
      draw,
      shape=regular polygon,
      regular polygon sides=3,
      inner sep=0pt,
      outer sep=0pt,
      \forestoption{node options},
      anchor=\forestoption{anchor}
      ]
      (\forestoption{name}) {\foresteoption{content format}};
    },
    child anchor=parent,
  },
  downsideTria/.style={
     node format={
      \noexpand\node [
      draw,
      shape=regular polygon,
      regular polygon sides=3,
      shape border rotate=180,
	  inner sep=0pt,
      outer sep=0pt,
      \forestoption{node options},
      anchor=\forestoption{anchor}
      ]
      (\forestoption{name}) {\foresteoption{content format}};
    },
    child anchor=parent, 
  },
  downsideTriaTerm/.style={
     node format={
      \noexpand\node [
      draw,
      shape=regular polygon,
      regular polygon sides=3,
      shape border rotate=180,
	  inner sep=0pt,
	  outer sep=0pt,
	  fill=green,
      \forestoption{node options},
      anchor=\forestoption{anchor}
      ]
      (\forestoption{name}) {\foresteoption{content format}};
    },
    child anchor=parent, 
  },
  upsideTriaTerm/.style={
    node format={
      \noexpand\node [
      draw,
      shape=regular polygon,
      regular polygon sides=3,
      inner sep=0pt,
      outer sep=0pt,
	  fill=green,
      \forestoption{node options},
      anchor=\forestoption{anchor}
      ]
      (\forestoption{name}) {\foresteoption{content format}};
    },
    child anchor=parent,
  },
  upsideTriaYellow/.style={
    node format={
      \noexpand\node [
      draw,
      shape=regular polygon,
      regular polygon sides=3,
      inner sep=0pt,
      outer sep=0pt,
	  fill=yellow,
      \forestoption{node options},
      anchor=\forestoption{anchor}
      ]
      (\forestoption{name}) {\foresteoption{content format}};
    },
    child anchor=parent,
  },
  downsideTriaYellow/.style={
     node format={
      \noexpand\node [
      draw,
      shape=regular polygon,
      regular polygon sides=3,
      shape border rotate=180,
	  inner sep=0pt,
	  outer sep=0pt,
	  fill=yellow,
      \forestoption{node options},
      anchor=\forestoption{anchor}
      ]
      (\forestoption{name}) {\foresteoption{content format}};
    },
	child anchor=parent,
  }
}

\tikzset{
myedge/.style={
  decoration={
   markings,
   mark=at position 0.5 with \node {\usebox\mybox};
  },
  postaction=decorate
  }
}
\addbibresource{literatur.bib}


\begin{document}


% ----------------------------------------------------------------------------------------------------------
% Titelseite
% ----------------------------------------------------------------------------------------------------------
\newcommand{\studierenderName}{Korbinian Federholzner}
\student{\studierenderName}		% Studierender
{3114621}						% Matrikelnummer
{Techniche Informatik}			% Studiengang

\MyTitelseite{}	% Optionales Logo des extern betreuenden Unternehmens
{1}								% Style der Titelseite (1 oder 2)
{Bachelorarbeit}				% Typ der Abschlussarbeit (\in {Bachelorarbeit, Masterarbeit})
{Entwicklung einer künstlichen Intelligenz für Brettspiele und deren Anbindung an eine Touch-Hardware über mobiele Endgeräte}				% Thema der Arbeit						
{Prof.\ Dr.\ Carsten Kern}		% Betreuer
{Prof.\ Dr. \ Daniel Jobst}	% Zweitgutachter
{31.08.\the\year}				% Abgabedatum

\thispagestyle{empty}
~\pagebreak

\setcounter{page}{1} 

% ----------------------------------------------------------------------------------------------------------
% Eigensctändigkeitserklaerung
% ----------------------------------------------------------------------------------------------------------
\include{inhalt/erklaerung}

% ----------------------------------------------------------------------------------------------------------
% Abstract
% ----------------------------------------------------------------------------------------------------------
\thispagestyle{empty}
\setstretch{1.15} % Zeilenspacing
\section*{Zusammenfassung}

\bigskip 


Diese Arbeit befasst sich mit der Erweiterung, der Software ReversiXT um eine Dame Künstliche Intelligenz (KI). 
Dabei wird die Anwendung um eine Verbindungsmöglichkeit für Mobile Endgeräte erweitert, welche es erlaubt, eine Verbindung
mit der Hardware herzustellen und die KI herauszufodern.
\\
Als KI Algorithmen sind MCTS und Minimax, mit Verbesserungen, wie Alpha-Beta Pruning und Zugsortierung Implementiert.
Welche im Laufe der Arbeit vorgestellt und zum Schluss, anhand einer Simulation, auf ihre Spielstärke gestest werden.
Die Implementierung der Schnittstellen, sowie Softwarekomponenten, ist Modular gehalten, was die Anwendung 
leicht um weitere Spiele, sowie KI-Algorithmen erweiterbar macht. 


% ----------------------------------------------------------------------------------------------------------
% Inhaltsverzeichnis
% ----------------------------------------------------------------------------------------------------------
\tableofcontents
\pagebreak

% ----------------------------------------------------------------------------------------------------------
% Abbildungsverzeichnis
% ----------------------------------------------------------------------------------------------------------
\lhead{}
\rhead{Abbildungsverzeichnis}
\listoffigures
\pagebreak

% ----------------------------------------------------------------------------------------------------------
% Tabellenverzeichnis (optional)
% ----------------------------------------------------------------------------------------------------------
\lhead{}
\rhead{Tabellenverzeichnis}
\listoftables
\pagebreak

% ----------------------------------------------------------------------------------------------------------
% Listingsverzeichnis (optional; Code nur, wenn wirklich sinnvoll und wichtig)
% ----------------------------------------------------------------------------------------------------------
%\lhead{}
%\rhead{Quellcodeverzeichnis}
%\lstlistoflistings
%\pagebreak

% ----------------------------------------------------------------------------------------------------------
% Abkürzungsverzeichnis (optional)
% ----------------------------------------------------------------------------------------------------------
\lhead{}
\rhead{Abkürzungsverzeichnis}
%\listoftables
\section{Abkürzungsverzeichnis}
\begin{acronym}[KDE]
\acro{BA}[BA]{Bachelorarbeit}
\acro{MA}[MA]{Masterarbeit}
\end{acronym}
\pagebreak


% ----------------------------------------------------------------------------------------------------------
% Inhalt
% ----------------------------------------------------------------------------------------------------------
% Abstände Überschrift
\titlespacing{\section}{0pt}{12pt plus 4pt minus 2pt}{8pt plus 2pt minus 2pt}
\titlespacing{\subsection}{0pt}{12pt plus 4pt minus 2pt}{6pt plus 2pt minus 2pt}
\titlespacing{\subsubsection}{0pt}{12pt plus 4pt minus 2pt}{4pt plus 2pt minus 2pt}

% Kopfzeile
\renewcommand{\sectionmark}[1]{\markright{#1}}
\renewcommand{\subsectionmark}[1]{}
\renewcommand{\subsubsectionmark}[1]{}
\lhead{Kapitel \thesection}
\rhead{\rightmark}

%\onehalfspacing
\setstretch{1.15}
\renewcommand{\thesection}{\arabic{section}}
\renewcommand{\theHsection}{\arabic{section}}
\setcounter{section}{0}
\pagenumbering{arabic}
\setcounter{page}{1}

% ----------------------------------------------------------------------------------
% Kapitel: Einleitung
% ----------------------------------------------------------------------------------
\section{Einleitung}
Das Thema dieser Arbeit ist die Implementierung einer künstlichen Intelligenz für diverse
Brettspiele. Dabei soll die resultierende Software auf einem von einem Raspberry Pi gesteuerten 
Touch-Bildschirm laufen, durch welchen ein Benutzer die künstliche Intelligenz herausfordern kann.
Außerdem gibt es die Möglichkeit, sich mit der Hardware über ein mobiles Endgerät wie einem 
Smartphone zu verbinden, um die KI oder den Spieler, der den Touch-Bildschirm bedient,
herauszufodern.

\subsection{Motivation}
Das Feld der künstlichen Intelligenz ist momentan eines der sich am schnellsten 
entwickelnden Felder der Informatik. Dabei spielt die Spieltheorie schon seit Anfang eine 
große Rolle. So bieten klassische Brettspiele wie z. B. Schach, Dame oder Mühle nicht nur 
eine klar definierte Abstraktion von Problemen der realen Welt, sondern sie können auch von
dem Großteil der Bevölkerung verstanden und gespielt werden. Das Meistern einer dieser 
Brettspiele wird auch oft mit hohem Grad an Intelligenz gleichgesetzt. Viele Algorithmen,
die in der Spieltheorie entwickelt wurden, haben sich auch erfolgreich auf andere Felder
der Informatik übertragen lassen. Ebenso hat sich die Art, wie Brettspiele gespielt werden, 
durch das Verwenden von künstlicher Intelligenz auch verändert, da die KI Züge in Betracht 
zieht, die auf den ersten Blick recht ungewöhnlich und nachteilhaft aussehen, sich aber als 
extrem stark herausstellen. Diese Arbeit versucht eine künstliche Intelligenz für das
Brettspiele Dame zu implementieren.

\subsection{Aufgabenstellung}
\label{chap:Aufgabenstellung}
Im Rahmen der Bachelorarbeit soll eine künstliche Intelligenz entwickelt werden, welche 
das Brettspiele Dame spielen kann. Gegeben ist eine Software, bei welcher man in der Lage ist,
das Spiel ReversiXT (Reversi Extreme) gegen eine KI, sowie sich selbst zu spielen. Diese 
Software soll um einen Game Server, die KI und das neue Spiel in der GUI, erweitert werden.
Außerdem soll ein Benutzer in der Lage sein, sich mit seinem Smarthphone mit der Hardware zu 
verbinden, um neue, sowie die alte KI herausfordern zu können.

\subsection{Struktur dieser Arbeit}
Die Arbeit ist folgendermaßen aufgebaut: \\
Kapitel 2 befasst sich mit den Grundlagen zu Dame, sowie den verwendeten 
künstliche Intelligenz Algorithmen. Dabei werden zu Dame auch die Grundregeln der
verwendeten Variante erklärt. Bei den KI Algorithmen handelt es sich um die Algorithmen, die in 
der Arbeit verwendet und miteinander verglichen werden. \\
In Kapitel 3 werden die Anforderungen, welche gefordert sind, vorgestellt.

\pagebreak
\section{Grundlagen}
\label{chap:Grundlagen}
Dieses Kapitel gibt einen Überblick über die theoretischen Grundlagen, die für das Verständnis 
dieser Arbeit notwendig sind. Zunächst werden die Grundregeln des behandelten Brettspieles 
vorgestellt. Darauf folgt eine Erklärung der künstlichen Intelligenz Algorithmen, welche für 
folgenden Kapitel von großer Relevanz sind. \cite{KuenstlicheIntelligenzNorvig}

%\subsection{Brettspiele}
\subsection{Dame}
Dame ist eines der ältesten Brettspiele, wobei erste Varianten 3000 v. Chr. im irakischen Ur
entdeckt wurden, welche sich jedoch stark vom Moderenen Dame unterscheiden. Das Dame so wie es Heutzutage 
bekannt ist, stammt vermutlich vom Spiel Alquerque ab. Da damals nicht jeder ein Spielbrett für Alquerque 
hatte, wurden die verfügbaren Schachbretter mit den Backgammon Steinen verwendet \cite{SpieleDerWelt}.
Dieses Spiel wurde später mit Regeln wie Schlagzwang erweitert, was zur Entstehung des Heutigen Dame führte.
Auch in der Heutigen Zeit gibt es noch hunderte Varianten von Dame welche sich Weltweit unterscheiden.

\subsubsection{Varianten des Spieles}
Das Spiel Dame so wie es im Deutschsprachigen Raum gespielt wird, findet auf einem 8x8 Schachbrett statt. 
International wird jedoch ein 10x10-Brett und in der Kanadsichen Dame ein 12x12-Brett verwendet. 
Normalerweise können Figuren nur Diagonal Ziehen, jedoch gibt es 
auch Abwandlungen des Spieles wie bei Türkischer Dame, bei welcher Spielsteine horizontal und vertikal gezogen werden können.
Das Spiel Unterscheidet sich des weiteren durch die Fähigkeiten
des Dame-Spielsteins, welcher in einigen Varianten nur eine Feld in beliebige Richtung ziehen darf, in anderen 
So viel Felder wie möglich. Die Art wie geschlagen ist bei diesen Varianten ebenso anders, so erlauben
einige Varianten das Schalgen eines normalen Spielsteines in die Rückwertsrichtung, andere verbieten 
jedoch diesen Zug. Außerdem gibt es Spielformenen wie Baschni, eine Form der Russischen Dame bei welcher durch schlagen 
eines Spielesteines, dieser nicht vom Spiel entfernt wird sondern auf den Spielstein der ihn geschlagen hat gesetzt wird. 
Wird dieser ``Turm'' aus mehreren Spielsteinen geschlagen, so wird nur der oberste Stein entfernt. 
Bei der Variante, welche für diese Arbeit verwendet wird, handelt es sich um Internationale Dame, da diese 
Spielvariante im Vereinssport und Internationalen Wettbewerben verwendet wird. \cite{DraughtsHistory}

\subsubsection{Internationale Dame}
In dieser Variante des Spieles wird auf einem 10x10 Brett mit Schachbrett-Muster gespielt.
Siehe Abbildung \ref{fig:checkersboard}. Die Spielsteine
sind scheibenförmig und in zwei Farben vorhanden, meist schwarz und weiß und dürfen nur auf den dunklen
Feldern des Schachbrettes bewegt werden. Es gibt zwei Arten von Spielsteinen. Normale Spielsteine, welche
nur in Richtung des Gegners bewegt werden, aber Rückwerts schlagen dürfen, und Damen, welche
in alle Richtungen beliebig viele Felder fahren und schlagen dürfen. Allgemein herrscht Schlagzwang,
was bedeutet, dass falls ein Spieler die Möglichkeit hat zu schlagen, er auch schlagen muss.
Spielsteine die geschlagen wurden, werden aus dem Spiel entfernt und nicht wie bei anderen Varianten 
auf den Schlagenden Spielstein gestapelt.
Ein normaler Spielstein wird zur Dame, falls er in die hinterste Reihe des Gegners kommt.
Ziel des Spieles ist es, entweder alle Steine des Gegners zu schlagen, oder den Gegner in eine Situation
zu zwingen, in der er keine Züge mehr machen kann. \cite{InternationalCheckersRules}

\vspace{1em}
\begin{minipage}{\linewidth}
	\centering
	\includegraphics[width=0.5\linewidth]{pics/checkersboard.png}
	\captionof{figure}[DameSpielfeld]{ Das 10x10 Spielfeld aus der Anwendung }
	\label{fig:checkersboard}
\end{minipage}

\subsection{Die gegebene ReversiXT Applikation}
Wie zuvor im Kapitel \ref{chap:Aufgabenstellung} auf Seite \pageref{chap:Aufgabenstellung} erwähnt, ist das Ziel der Arbeit, das Erweiteren
der ReversiXT Software um das Spiel Dame. Diese Software ist eine auf Webtechnologie basierte Anwendung, welche es dem Benutzer erlaubt
das Spiel Reversi zu spielen. Startet man die die Anwendung zum ersten mal wird man vom Hauptmenü begrüßt, siehe Abbildung \ref{fig:ReversiMenu}.


\vspace{1em}
\begin{minipage}{\linewidth}
	\centering
	\includegraphics[width=0.7\linewidth]{pics/ReversiMenues.png}
	\captionof{figure}{ Das Hauptmenü der Anwendung}
	\label{fig:ReversiMenu}
\end{minipage}

Über dieses Menü kann man entweder den Bildschirmschoner, die Anleitungen, die Videos oder ein Spiel starten. Der Bildschirmschoner 
zeigt Spiele bei, welchen zwei KIs gegeneinander Spielen, oder Videos welche in der Video Kategorie gespeichert sind. Wird Spielen aufgerufen, 
so wird man auf eine Seite weitergeleitet, welche ein Auswahlmenü zum starten eines Reversi Spieles bietet, siehe \ref{fig:ReversiGameSelection}.

\vspace{1em}
\begin{minipage}{\linewidth}
	\centering
	\includegraphics[width=0.7\linewidth]{pics/ReversiGameSelection.png}
	\captionof{figure}{ Das Auswahlmenü eines Reversi Spieles}
	\label{fig:ReversiGameSelection}
\end{minipage}

In diesem Menü kann der Benutzer aus verschiedenen Karten und Spielern ein Spiel auswählen und starten. Jede Karte hat eine Anzahl an Spieler 
in Klammern dahinert stehen, so hat ``bomb Test (8p)'' eine Anzahl von 8 Spielern zur Auswahl. Die Auswahl der Spieler unterscheidet hierbei
zwischen ``Mensch'', den Benutzer und ``Einfache KI'', eine Einfach implementierte Reversi KI. Wird ``Spiel starten'' gedrückt, wird
ein neues Spiel Reversi gestartet und man wird auf eine Seite, die das Spielbrett darstellt weitergeleitet, siehe Abbildung \ref{fig:ReversiGame}.

\vspace{1em}
\begin{minipage}{\linewidth}
	\centering
	\includegraphics[width=0.7\linewidth]{pics/ReversiGame.png}
	\captionof{figure}{ Ein gestartetes Reversi Spiele }
	\label{fig:ReversiGame}
\end{minipage}

Die Seite des Spieles ist unterteilt in ein Informationsmenü, welches informationen zum laufenden Spiel darstellt und dem Spielbrett mit dem
Momentanzustand des Spieles. Falls der Benutzer an der Reihe ist, kann durch klicken auf das Spielfeld ein neuer Zug ausgeführt werden.
Des weiteren bietet das Infomenü eine Möglichkeit das Spiel neuzustarten oder zu beenden. 
Hat ein Spieler gewonnen, wird ein Info-Fenster eingeblendet, welches den Benutzer über den Sieger informiert.


\subsection{Künstliche Intelligenz Algorithemen}
Um ein Brettspiel wie Dame als künstliche Intelligenz abzubilden, gibt es viele mögliche Algorithmen.
Diese reichen von Maschinelearning bis zu in dieser Arbeit verwendeten Baum-Algorithmen.
Baum-Algorithemen bei Brettspielen zeichnen sich dadurch aus, dass diese einen Spielbrettzustand als Knoten abbildet.
Dabei hat jeder Knoten einen oder mehrere Nachfolgerknoten welche wiederum alle Züge, die aus seiner Stellung gespielt werden können, darstellen.
Die Schwirigkeit hierbei ist die Auswahl des besten Pfades zum Sieg und dabei zu beachten, dass die Algorithmen so wenig wie möglich 
Rechenzeit dafür brauchen.

\subsubsection{Minimax}
\label{chap:Minimax}
Der Minimax Algorithmus wird verwendet, um einen optimalen Spielzug in Spielen mit perfekter Information, bei zwei Spielern
zu finden. Dazu wird eine Baumstruktur verwendet, welche den Zustand des Spielbrettes als Knoten hat, siehe Abbildung \ref{fig:minimax}.
Alle Züge, die von einer Stellung aus möglich sind, werden in den Kindknoten des jeweiligen Knotens gespeichert.
Der Wurzelknoten beschreibt den momentanen Zustand des Spieles bei dem der Algorithmus aufgerufen wird. Die Blattknoten am Ende des Baumes
entsprechen entweder einer Stellung in der das Spiel beendet wurde, oder der Stellung bei einer Tiefe, bei der der Algorithmus 
aufgehört hat zu suchen. Die Angabe einer Tiefe ist nötig,
da Spiele wie Schach oder Dame einen extrem großen Suchbaum zur Folge hätten und das Suchen eines Endzustandes in diesen sehr viel Zeit
beansprucht. \cite{MinimaxComparison}


Zum Beispiel kann die Abbildung \ref{fig:minimax} durch den Baum aus der Grafik \ref{fig:value}, mit einer Suchtiefe von vier, dargestellt werden.
Dabei wird jeder Terminal-Knoten, ein Knoten bei dem das Spiel vorbei ist oder die Endtiefe erreicht wurde (im Beispiel Grün makiert), 
mit einer Bewertungsfunktion bewertet:
\begin{itemize}
    \item Normale Figur: +1 für Weiß und -1 für Schwarz
    \item Dame: +3 für Weiß und -3 für Schwarz
    \item Spielende: $\infty$ für Weiß und $-\infty$ für Schwarz
\end{itemize} 
Ein Dreieck mit der lange Seite nach unten, steht für eine Maximierung der Kindknotenwerte, das andere Dreieck für eine
Minimierung. Die Bewertungen in den Endzuständen werden nach oben durchgereicht und je nachdem, ob der Elternknoten ein Maximierer oder ein
Minimierer ist, bekommt er einen neuen Wert zugewiesen.
Im Beispiel kann man sehen, dass egal welche Züge gewählt werden es immer zu einem Materialverlust von Weiß kommt. Würde Weiß
die zweite Option wählen, so ist das Spiel nach dem Nächsten Zug von Schwarz schon entschieden und Weiß verliert. 


\begin{figure}[H]
\centering
\scalebox{0.7}
{%
\begin{forest}
[{\includegraphics[scale=0.2]{pics/root.png}}
    [{\includegraphics[scale=0.15]{pics/1goodmove.png}}
        [{\includegraphics[scale=0.15]{pics/21goodmove.png}}
            [{\includegraphics[scale=0.12]{pics/311goodmove.png}}
                [\vdots]
            ]
            [{\includegraphics[scale=0.12]{pics/312goodmove.png}}
                [\vdots]
            ]
            [{\includegraphics[scale=0.12]{pics/313goodmove.png}}
                [\vdots]
            ]
        ]
        [{\includegraphics[scale=0.15]{pics/22badmove.png}}
            [{\includegraphics[scale=0.12]{pics/322badmove.png}}]
            [{\includegraphics[scale=0.12]{pics/321badmove.png}}
                [\vdots]
            ]
        ]
    ]
    [{\includegraphics[scale=0.15]{pics/1badmove.png}}
        [{\includegraphics[scale=0.15]{pics/2winAfter1badmove.png}}]
    ]
]
\end{forest}
}
\caption{Minimax-Baum mit Spielfeldzustand als Knoten}
\label{fig:minimax}
\end{figure}

\begin{figure}[H]
\centering
{%
\begin{forest}
    [-1, upsideTria
        [-1, downsideTria
            [-1, upsideTria
                [-1, downsideTria
                    [-1, upsideTriaTerm]
                ]
                [-1, downsideTria
                    [-1, upsideTriaTerm]
                ]
                [-1, downsideTria
                    [-1, upsideTriaTerm]
                ]
            ]
            [$\infty$, upsideTria
                [$\infty$, downsideTriaTerm]
                [-1, downsideTria
                    [-1, upsideTriaTerm]
                    [-1, upsideTriaTerm]
                ]
            ]
        ] 
        [{\fontsize{9}{8}\selectfont $-\infty$}, downsideTria
            [{\fontsize{9}{8}\selectfont $-\infty$}, upsideTriaTerm]
        ] 
    ]
\end{forest}
}
\caption{Minimax-Baum mit Bewertung der Stellungen}
\label{fig:value}
\end{figure}


%\vspace{1em}
%\begin{minipage}{\linewidth}
%	\centering
%	\includegraphics[width=0.5\linewidth]{pics/minimax.png}
%	\captionof{figure}[minimax]{ Beispiel eines Minimax Suchbaumes }
%	\label{fig:minimax}
%\end{minipage}

\subsubsection{Iterative Deepening}
\label{chap:IterativeDeepening}
In komplexeren Spielen wie Go, Schach oder Dame ist es wegen des Rechenaufwands sehr schwer, den kompletten Baum von Minimax bis zu den
Endzuständen aufzubauen. Deswegen wird in diesen Spielen der Baum nur bis zu einer gewissen Tiefe aufgebaut. Da es aber bei einer gleichen
Tiefe für verschiedene Stellungen zu unterschiedlichen Dauern der Suche kommen kann, ist es problematisch eine fixe Suchtiefe anzugeben, 
vor allem wenn mit Zeitlimits gearbeitet wird. Iterative Deepening hilft hierbei, der Ablauf des Algorithmus ist wie folgt:
Zuerst führe Minimax für eine Tiefe von eins aus. Danach, verwerfe alle generierten Knoten des Baumes und starte erneut von Anfang, aber dieses
Mal bis zu einer Tiefe von zwei. Dieses verwerfen und neu starteten wird so oft wiederholt bis ein Zeitlimit erreicht wird, siehe Abbildung \ref{fig:IterativeDeepening}. 
Der letzte aufgebaute Baum, bevor neugestartet wird, wird zwischengespeichert und falls Minimax bis zum Ablauf des Zeitlimits nicht fertig ist,
wird die momentane Berechnung verworfen und der letzte gespeicherte Baum verwendet. Ein Nachteil dieser Methode ist, dass 
der Rechenaufwand der ersten Tiefen verschwendet wird, da diese Ergebnisse verworfen werden. Jedoch beeinflusst diese 
verschwendete Rechenzeit nicht die asymptotische Laufzeit des Algorithmus, da die meiste Arbeit in der untersten Tiefe der 
Suche gebraucht wird. \cite{IterativeDeepening}.

\begin{figure}[H]
\centering
\begin{tabular}{*{6}{|T}}
    Tiefe 1 & Tiefe 2 & Tiefe 3 \\
    \begin{forest}
        [$\triangle$
            [$\nabla$] 
            [$\nabla$] 
        ]
    \end{forest}
    &
    \begin{forest}
        [$\triangle$
            [$\nabla$ 
                [$\triangle$]
                [$\triangle$]
            ]
            [$\nabla$ 
                [$\triangle$]
            ]
        ]
    \end{forest}
    &
    \begin{forest}
        [$\triangle$
            [$\nabla$ 
                [$\triangle$
                    [$\nabla$]
                ]
                [$\triangle$
                    [$\nabla$]
                    [$\nabla$]
                ]
            ]
            [$\nabla$ 
                [$\triangle$
                    [$\nabla$]
                    [$\nabla$]
                    [$\nabla$]
                ]
            ]
        ]
    \end{forest}
    \\
\end{tabular}
\caption{Ablauf des Iterativen Deepenings}
\label{fig:IterativeDeepening}
\end{figure}

\subsubsection{Alpha-Beta Pruning}
\label{chap:alphaBeta}
Das Alpha-Beta Pruning ist eine Optimierung zum Minimax Algorithmus. Die Idee des Algorithmus ist, 
dass manche Zweige des Suchbaums nicht untersucht werden müssen, da für den anderen Spieler diese
Züge nicht in Frage kommen. Hierbei ist $\alpha$ der Wert für den Spieler, für den die niedrigen Werte 
besser sind und $\beta$ für den anderen Spieler. Für jeden Knoten, je nachdem, ob er ein maximierender
oder ein minimierender Knoten ist, wird überprüft, ob ein Kind-Knoten, welcher einen neuen Wert
erhalten hat, nicht mehr vom Knoten beachtet werden muss. Der Vorteil des Alpha-Beta Prunings zu Minimax ist, 
dass der verbrauchte Speicher weniger wird, da vom Baum Zweige nicht beachtet werden müssen.
Was wiederum zur Folge hat das die Ausführungszeit des Algorithmus schneller ist und gleichzeitig auch dasselbe 
Ergbnis wie Minimax zur Folge hat.
\cite{AlphaBeta}

Wenn man Abbildung \ref{fig:minimax} und \ref{fig:value} als Beispiel nimmt und den Alpha-Beta Pruning Algorithmus anwendet,
so kann der Zweig mit dem Wert $\infty$ ignoriert werden, siehe Abbildung \ref{fig:AlphaBeta}. Der gelbe Knoten bekommt eine -1 
durch seinen linken Zweig vorübergehend zugewiesen. Da der Knoten des Rechten Zweiges ein Maximierer ist, also immer den Wert des
Kindknotens mit dem höchsten Wert nimmt, und dieser bereits einen Knoten mit dem Wert $\infty$ gefunden hat, wird sein Wert 
definitiv mindestens $\infty$ sein. Der restliche rechte Zweig des gelben Knotens kann nun ignoriert werden, da der linke Zweig mit
-1 definitiv kleiner sein wird.

\begin{figure}[H]
\centering
{%
\begin{forest}
    [-1 , upsideTria
        [-1, downsideTriaYellow
            [-1, upsideTria
                [-1, downsideTria
                    [-1, upsideTriaTerm]
                ]
                [-1, downsideTria
                    [-1, upsideTriaTerm]
                ]
                [-1, downsideTria
                    [-1, upsideTriaTerm]
                ]
            ]
            [$\infty$, upsideTria, edge={myedge}
                [$\infty$, downsideTriaTerm]
                [?, downsideTria]
            ]
        ] 
        [{\fontsize{9}{8}\selectfont $-\infty$}, downsideTria
            [{\fontsize{9}{8}\selectfont $-\infty$}, upsideTriaTerm]
        ] 
    ]
\end{forest}
}
\caption{Gewinn durch Alpha-Beta Pruning}
\label{fig:AlphaBeta}
\end{figure}

\subsubsection{Zugsortierung}
\label{chap:Zugsortierung}
Zugsortierung ist eine Erweiterung zur Alpha-Beta Suche.
Da Alpha-Beta Pruning abhängig von der Reihenfolge, in der die Zustände untersucht werden, ist es sinnvoll
die Nachfolger zu wählen, welche die besten Werte erbringen. Den besten Nachfolger findet man, in dem man eine
weitere Bewertungsfunktion einbaut, die nicht so genau wie die Bewertungsfunktion an den Terminalknoten sein muss. 
Wenn ein Knoten also alle möglichen Nachfolgezüge als Kinder bekommt, werden auf diese die vereinfachte Bewertungsfunktion
angewandt und je nach Ergebnis der Funktion werden die Nachfolger sortiert. Dadurch, dass der beste Zug nun
sehr weit am Anfang steht, ist es sehr wahrscheinlich, dass die anderen Züge durch Alpha-Beta Pruning ignoriert werden.
\cite{KuenstlicheIntelligenzNorvig}

Auf der Linken Seite der Abbildung \ref{fig:Sorting} kann man erkennen, dass Alpha-Beta Pruning keinen Effekt 
auf die beiden gelben Knoten hätte. Ändert man jedoch die Reihenfolge der Knoten, so kann der Knoten mit dem Wert 42
ignoriert werden. Bei der Bewertungsfunktion könnte man die bedrohte Figuren als Faktor haben, um auf dieses Ergbnis zu kommen.


\begin{figure}[H]
\centering
{%
\begin{forest}
    [-1, upsideTria
        [-1, downsideTria
            [42, upsideTriaYellow
                [42, downsideTriaTerm]
                [-1, downsideTria
                    [-1, upsideTriaTerm]
                    [-1, upsideTriaTerm]
                ]
            ]
            [-1, upsideTriaYellow
                [-1, downsideTria
                    [-1, upsideTriaTerm]
                ]
                [-1, downsideTria
                    [-1, upsideTriaTerm]
                ]
                [-1, downsideTria
                    [-1, upsideTriaTerm]
                ]
            ]
        ] 
        [{\fontsize{9}{8}\selectfont -44}, downsideTria
            [{\fontsize{9}{8}\selectfont -44}, upsideTriaTerm]
        ] 
    ]
\end{forest}
\raisebox{1\height}{\includegraphics{pics/ArrowRight.png}}
\begin{forest}
    [-1 , upsideTria
        [-1, downsideTria
            [-1, upsideTriaYellow
                [-1, downsideTria
                    [-1, upsideTriaTerm]
                ]
                [-1, downsideTria
                    [-1, upsideTriaTerm]
                ]
                [-1, downsideTria
                    [-1, upsideTriaTerm]
                ]
            ]
            [42, upsideTriaYellow, edge={myedge}
                [42, downsideTriaTerm]
                [?, downsideTria]
            ]
        ] 
        [{\fontsize{9}{8}\selectfont -44}, downsideTria
            [{\fontsize{9}{8}\selectfont -44}, upsideTriaTerm]
        ] 
    ]
\end{forest}
}
\caption{Beispiel einer Zugsortierung}
\label{fig:Sorting}
\end{figure}

\subsubsection{Monte Carlo Tree Search (MCTS)}
\label{chap:MCTS}
Der Monte Carlo Tree Search Algorithmus, ist ein heuristischer Algorithmus, bei welchem von einem
Zustand eines Spieles zufällig endlich viele Simulationen durchgeführt werden. Die Simulation endet, wenn
ein Ergebnis des simulierten Spieles feststeht. Das Wiederholen der Simulationen aus verschiedenen Knoten
hat zur Folge, dass das Ergebnis immer genauer wird. Am Ende wird der Knoten gewählt, bei dem die 
Simulationen die besten Ergebnisse für den momentanen Spieler gezeigt haben. Ein Vorteil des 
MCTS-Algorithmus gegenüber Minimax ist, dass erst am Ende eines Durchlaufs eine Bewertungsfunktion
benötigt wird. Allgemin besteht der Algorithmus aus vier Schritten:
\begin{itemize}
    \item \textit{Selektion}: Versucht wird, einen Zustand zu finden der noch erweiterbar ist, also
        einen Zustand zu finden, der kein Endzustand ist und noch nicht besuchte Züge hat.
    \item \textit{Expansion}: Der Spielbaum wird zufällig um einen noch nicht besuchten Zug erweitert.
    \item \textit{Simulation}: Von dem gewählten Knoten aus wird nun ein Spiel zufällig bis zum Ende 
        simuliert. 
    \item \textit{Backpropagation}: Das Ergebnis der Simulation wird den vorhergehenden Knoten mitgeteilt
        und diese werden mit diesem aktualisiert.
\end{itemize}
Da man im Normalfall nicht beliebig viel Zeit hat, alle Möglichkeiten zu simulieren, versucht man die limitierte Zeit
so gut wie möglich zu nutzen, um die richten Knoten zum Expandieren zu wählen. Dazu wird der \textit{upper confidence bound for trees} (UCT) verwendet.
Die UCT Formel lautet:
\begin{align}
	w + c \sqrt{\frac{\log{N}}{n}}
\end{align}
Wobei $w$ die prozentuale Anzahl an Gewinnen des Knotens, $N$ die Anzahl der gesamten Expansionen und $n$ die Expansionen nur an dem 
betrachteten Knoten sind. Die Aufgabe der UCT Formel ist das Erreichen von zwei im Konflikt stehenden Zielen. Das erste Ziel ist es die Knoten die bisher die 
höchsten Chancen auf den Gewinn haben, tiefer zu simulieren, um eine bessere Genauigkeit des besten Zuges zu haben.
Das zweite Ziel ist, Knoten die noch nicht sehr oft besucht worden sind genauer zu untersuchen, da diese vielversprechender sein könnten als gedacht.
Für die Balancierung der beiden Ziele gibt es den Parameter $c$ \cite{DeepLearingGo}.

Als Beispiel wird die zuvor verwendete Stellung aus \ref{fig:minimax} benutzt, aus welcher der MCTS-Baum in Abbildung \ref{fig:MCTSTree} entsteht.
"`B"' steht für die Siege aus Schwarzer und "`W"' für Siege aus Weißen Sicht, nach der Beendigung von 33 Simulationen.  
Weiß entscheidet sich in diesem Baum für den gelben Knoten, da dieser bei Betrachtung der Gewichtung von Siege eine höhere
Gewinnchance für ihn hat.

\begin{figure}[H]
\centering
{%
\begin{forest}
    for tree={%
        edge={->},
    }
    [B:27 W:6, circle, draw
        [B:19 W:6, circle, fill=yellow, draw
            [B:12 W:1, circle, draw
                [B:4 W:1, circle, draw
                    [B:4 W:1, circle, draw [{\vdots}]]
                ]
                [B:3 W:0, circle, draw
                    [B:3 W:0, circle, draw [{\vdots}]]
                ]
                [B:5 W:0, circle, draw
                    [B:5 W:0, circle, draw [{\vdots}]]
                ]
            ]
            [B:7 W:5, circle, draw
                [B:0 W:4, circle, draw]
                [B:7 W:1, circle, draw
                    [B:3 W:1, circle, draw [{\vdots}]]
                    [B:4 W:0, circle, draw [{\vdots}]]
                ]
            ]
        ] 
        [B:8 W:0, circle, draw
            [B:8 W:0, circle, draw]
        ] 
    ]
\end{forest}
}
\caption{Minimax-Baum mit Bewertung der Stellungen}
\label{fig:MCTSTree}
\end{figure}



\pagebreak
% ----------------------------------------------------------------------------------
% Kapitel: ???
% ----------------------------------------------------------------------------------

\section{Architektur der Software}
Dieses Kapitel setzt sich mit der Softwarearchitktur auseinander. 
%Die Architektur ist nach dem Model der Client-Server-Architektur entworfen. 
%Die Client-Server-Architektur trennt die Software in drei
%Teile, zwei Clients und einen Server: \\
%Der Gameserver, an den neue Züge geschickt werden und welcher dan Momentanzustand des Spielbrettes hält, dient als Server.
%Die ReversiXT GUI, über welche der User Spiele starten und KI's auswählen kann und dem KI Client, welcher eine künstliche Intelligenz für Dame ist, 
%sind die beiden Clients die sich mit dem Server verbinden.
\subsection{Überblick}
Dadurch, dass es sich bei der ReversiXT Software um ein Projekt handelt, welches bereits eine Implementierung einer Reversi Spieleoberfläche besitzt und 
um Dame erweitert werden soll, muss die Software Modular und Erweiterbar genug sein, um durch weitere Spiele erweitert werden zu können.
Die Vorraussetzungen um ein neues Spiel hinzuzufügen, ist das Vorhanden sein eines Gameservers, welcher die Spiellogik, 
den Zustand, sowie die Zugreihenfolge des Spieles verwaltet. Desweiteren wird ein Gameclient benötigt, welcher die KI-Logik zur Berechnung
neuer Züge bereitstellt. Bei diesen beiden Komponenten handelt es sich um eigene Anwendungen, was den Vorteil hat, dass sie einfach 
durch neue Implementierungen ausgetauscht werden können. Möchte man das Spiel Dame um extra Regeln erweitern, welche in anderen Ländern gespielt werden
und das umschreiben des alten Gameservers vermeiden, so kann man ihn einfach durch einen neuen ersetzten, welcher das gleiche Protokoll der Kommunikation 
unterstützt, jedoch eine andere Implementierung aufweist. Gleiches gilt auch den Gameclient, bei welchem die Softwarekomponente durch 
eine andere KI ersetzt werden kann, welche zum Beispiel Maschinelearning anstatt Graphalgorithmen unterstüzt. 

In Abbildung \ref{fig:DeploymentDiagram} ist das Projek als Verteilungsdiagramm dargestellt. 
Neben den Oben bereits erwähnen Gameserver und Gameclient, handelt es sich bei der Webapp um den Teil der Anwendung, der das Starten und Stoppen der Spiele verwaltet.
Der Frontend Teil der Webapp übernimmt die Darstellung der Graphischen Oberfläche der Website. Das Backend ist für die Kommunikation zu den anderen 
Bestandteilen verantwortlich. Außerdem beeinhaltet die Software einen Reverse Proxy, welcher wegen der Anbindung externer Mobilegeräte verwendet wird.

\vspace{1em}
\begin{minipage}{\linewidth}
	\centering
	\includegraphics[width=1.0\linewidth]{pics/DeploymentDiagram.png}
	\captionof{figure}{ Die Anwendung als Verteilungsdiagramm}
	\label{fig:DeploymentDiagram}
\end{minipage}



\subsection{Netzwerkkommunikation}
Durch das Aufteilen der Anwendung in mehrere ausführbare Programme, teilen sich die Programme keinen gemeinsamen Quellcode und laufen parallel zueinander. 
Dies hat zur Folge, das eine Netzwerkskommunikation zwischen den Programmen erforderlich ist, um den Ablauf zu regulieren.

\subsubsection{Kommunikation zwischen den Software-Teilen}
Die Kommunikation zwischen den drei Anwendungen ist in Abbildung \ref{fig:ComponentDiagram} auf Seite \pageref{fig:ComponentDiagram} als 
Komponentendiagramm dargestellt. Dabei ist die gelbe Komponente der Gameserver, die blaue die Webapp, die rote der Gameclient und die grüne der Reverse Proxy.
Der dunkelblaue eingefärbte Teil der Webapp, beeinhaltet das Frontend, also die Graphische Oberfläche (GUI) der Anwendung.
Die hellblauen Komponenten, beschreiben das Backend. 

In der Kommunikation zwischen den Backend und dem Frontend der Webapp Komponente, werden Züge über das ``incomingMove'' Interface des Backends an das Frontend
weitergereicht. Züge die von der Grafischen Oberfläche aus gespielt sind, kommen über das Interface ``sendInstructions'' an das Backend. Dieses
Interface wird außerdem verwendet um Befehle wie starten und stoppen eines Spieles zu übermitteln. Die Webapp Komponente hat zwei Ports nach Außen, 
welche Nachrichten an das Backend
weiterreichen. Der erste Port übernimmt alle Nachrichten, die mit Zügen in Verbindung stehen. So werden alle Züge die vom Gameserver angenommen werden
und als Broadcast an alle Teilnehmer weitergegeben werden, sowie die Züge die aus dem Frontend heraus gespielt über diesen Port übertragen.
Der zweite Port übernimmt das starten und stoppen des Gameservers und des Gameclients. 
Neben den Oben genannen Verbindungen des Gameservers, hat dieser außerdem eine Verbindung zum Gameclient. Über diese Verbindung werden Züge, die von der 
KI des Gameclients berechent werden an den Gameserver gesendet.

Will sich ein Smartphone mit der Anwndung verbinden, so werden alle Anfragen an Frontend oder Backend der Webapp vom Reverse Proxy abgefangen und 
nach überprüfung an diese Schnittstellen weitergegeben.


\vspace{1em}
\begin{minipage}{\linewidth}
	\centering
	\includegraphics[width=1.0\linewidth]{pics/Komponentendiagram.png}
	\captionof{figure}[Auswahlmenü]{ Die gesamte Anwendung im UML Komponentendiagram}
	\label{fig:ComponentDiagram}
\end{minipage}



\subsubsection{Kommunikation abhängig vom gewählten Spielmodus}
\label{chap:Ablauf}
Da die Applikation mit mehreren Spielmodi ausgestattet ist, ändert sich die Kommunikation und welche Komponente gestartet werden muss, abhängig vom Modus.
Die verfügbaren Spielmodi sind:
\begin{itemize}
    \item Benutzer gegen Benutzer
    \item Benutzer gegen KI
    \item KI gegen KI
\end{itemize}
Egal welche Option gewählt wird, der Gameserver wird immer gestartet, da dieser die Züge überprüft und Sieg oder Niederlage auswertet.
Bei Benutzer gegen Benutzer wird die KI Komponente der Software nicht gestartet, es kommuniziert die GUI-Komponente direkt mit Gameserver.
Wird sich für zwei KI Clients die gegeneinander spielen entschieden, so werden auch zwei gestartet.
Die Kommunikation findet nur mehr von Gameserver mit den beiden KI-Clients statt, jedoch hat die GUI-Komponente eine Man-in-the-Middle-Funktion
wodurch sie die Kommunikation abhört und die gespielten Züge darstellen kann.

Abbildung \ref{fig:UserVsUserSequenceDiagram} stellt ein UML Sequenzdiagramm dar, bei welchem ein Benutzer gegen die KI spielt.
Zuerst startet der Benutzer über die GUI mit dem Spielmodus ein Spiel, dadurch wird der Gameserver gestartet, sowie ein
Gameclient. Würde man stattdessen KI gegen KI als Parameter mitsenden würden zwei Gameclients gestartet werden. 
Der Gameserver wartet wärenddessen bis zwei Spieler bei ihm registriert haben. Ist die Registrierung abgeschlossen, werden Züge
abwechselnd von den Clients angefordert, bis ein Spieler gewonnen hat.

\vspace{1em}
\begin{minipage}{\linewidth}
	\centering
	\includegraphics[width=0.83\linewidth]{pics/UserVsAiCommunication.png}
	\captionof{figure}[SequenceDiagram]{ Das UML Sequenzdiagramm der Kommunikation von KI gegen Benutzer}
	\label{fig:UserVsUserSequenceDiagram}
\end{minipage}

\subsection{Dame Spiellogik}
\label{chap:Spiellogik}
Die Spiellogik für Dame ist in mehrfach in der Anwendung vorfindbar. Zum einen um Züge im KI-Client zu testen und zum anderen
für den Gameserver, zur Überprüfungen und Aktualisierung der Züge. In den Abbildungen \ref{fig:KIClientClassDiagram} auf Seite \pageref{fig:KIClientClassDiagram} 
und \ref{fig:GameServerClassDiagram} auf Seite \pageref{fig:GameServerClassDiagram}
befindet sich die Spiellogik im Paket \texttt{Checkers}. Diese Redundanz ist wichitg, da es sich beim Server und Client um eigenständige Prozesse handelt 
und diese nicht die gleiche Code-Basis haben. Eine Möglichkeit diese Redundanz zu vermeiden wäre, die Validierung in den Gameserver zu verschieben und 
dem Gameclient zu zwingen, dass jeder Zug von Gameserver validert werden muss. Dies erhöht aber die Netzwerkkommunikation und verlangsamt das finden eines Zuges, 
durch den daraus entstehenden Overhead.
\\
In Abbildung \ref{fig:GameLogic} zeigt die Spiellogik vom Paket \texttt{Checkers} unabhängig von der Applikation in der es verwendet wird.
Die Hauptklassen der Spiellogik sind, die \texttt{Board}-, \texttt{Move}- und \texttt{Player}-Klasse. Dabei hält die \texttt{Board}-Klasse die Information über das aktuelle Spielbrett, sowie 
Methoden um Spielsteine auf dem Spielbrett zu entfernen, oder zu bewegen. Die \texttt{Move}-Klasse dient zum repräsentieren eines Zuges, sie besteht aus dem 
Uhrsprungspunk, den Zielpunkt, dem Spieler der den Zug getätigt hat und um welchem Spielstein es sich handelt. 
\\
Die Klassen \texttt{CheckersForceCapture} und \texttt{CheckersAllPossibleMovesChecker}, sind Klassen zum finden von aller möglichen Züge die aus einer Stellung heraus gespielt werden
können. Dabei ist \texttt{CheckersForceCapture} verantwortlich für Züge bei denen ein Spieler gezwungen wird eine Figur zu schlagen und \texttt{CheckersAllPossibleMovesChecker} 
für alle Züge die keinen Schlagzwang haben. Die Aufteilung in zwei Klassen hat den Hintergrund, dass falls es sich um einen Schlagzwang handelt, braucht 
die \texttt{AllPossibleMoves}-Klasse nicht erzeugt werden um Züge zu finden. Um einen von der Netzwerkkommunikation eingehenden Zug zu überprüfen, 
werden die Metheoden der \texttt{CheckersMoveTypeCheck}-Klasse verwendet. Sie vergleichen einen gegebenen Zug, 
den zum Beispiel der Benuzter getätigt hat, mit den Zügen die aus \texttt{CheckersForceCapture} und \texttt{CheckersAllPossibleMovesChecker}
hervorgehen. Dieses Vergleichen ist sehr wichtig für den Gameserver, da alle Züge die von den Clients kommen valide sein müssen. Im Gameclient wird das Vergleichen nur 
als Extravalidierung verwendet um den gesendeten Zug aus dem Gameserver nochmal zu überprüfen und so Fehler zu minimieren.
\\
Das Anwenden eines Spielzuges auf das Spielbrett wird mittels der \texttt{MoveApplier}-Klasse im Gameclient und mit der \texttt{CheckersRules}-Klasse im Gameserver 
getätigt. Beide verfolgen im Prinzip das selbe Ziel, der Unterschied wird im Implementierungskapitel \ref{chap:Moves} erklärt.

\vspace{1em}
\begin{minipage}{\linewidth}
	\centering
	\includegraphics[width=1.0\linewidth]{pics/GameLogic.png}
	\captionof{figure}{ Das Paket Checkers welches die Spiellogik beeinhaltet }
	\label{fig:GameLogic}
\end{minipage}


\subsection{Gameserver}
Als Gameserver wird der Teil der Software betrachtet, welcher für das einhalten der Spielregeln und die Verwaltung der Spielfeldzustandes, 
verantwortlich ist. 

\subsubsection{Softwareaufbau des Gameservers}
Die Software des Gameservers ist in zwei Pakete aufgeteilt, in das Gamerulbook- und Serverconnection-Paket.
Der Serverconnection Teil ist für die Verbindung und das Dekodieren der Nachrichten der Clients verantwortlich.
Nachdem ein Client verbunden ist und seine Nachrichten schickt, wird diese dekodiert und falls das Protokoll korrekt eingehalten wird,
an den Gamerulbook Teil weitergeleitet. Das Gamerulbook Paket verwaltet den Spielzustand, sowie die Spielregeln des laufenden Spieles.
Es ist über das GameRules Interface mit dem Serverconnection Paket verbunden. Ein beliebiges Spiel kann mittels dieses Interfaces
als eigenes Paket festgelegt werden, was zu einer Möglichkeit führt den Gameserver um beliebig viele Spiele erweitern zu können.
Über den weiteren Inhalt des \texttt{Checkers} Paketes wird im vorherigen Kapitel \ref{chap:Spiellogik} berichtet.

Abbildung \ref{fig:GameServerClassDiagram} zeigt den Aufbau als UML-Klassendiagramm, dabei sind Dame und Mühle as Spiele über das
GameRules Interface als Pakete Beispielhaft gezeigt. 

\vspace{1em}
\begin{minipage}{\linewidth}
	\centering
	\includegraphics[width=1.0\linewidth]{pics/GameServerClassDiagram.png}
	\captionof{figure}[ClassDiagram]{ Das UML Klassendiagramm des Gameservers }
	\label{fig:GameServerClassDiagram}
\end{minipage}

\subsection{ReversiXT GUI}
Die ReversiXT Graphische Oberfläche (GUI) ist der Teil der Software der für die eigentliche Userinteraktion gedacht ist. User Können über die GUI
spiele starten, bei welchen sie gegen KI's oder andere User spielen können.

\subsubsection{Softwareaufbau der ReversiXT GUI}
Die ReversiXT GUI besteht aus zwei Teilen, dem Server und dem Webapp-Paket, siehe \ref{fig:ReversiXTGUIClassDiagram}. 
Dabei behandelt der Server, die Verbindung zu den anderen
Komponenten. Er startet den Gameserver und je nachdem wie viele KI's als Spieler gewählt werden, startet er auch die Gameclients dafür.
Sind die anderen Komponenten gestartet, kommunziert der Server der ReversiXT GUI nur noch mit Gameserver und benutzt dabei das Protokoll
welches dieser vorschreibt. Die Webapp behinhaltet das Benutzer Interface, über welches die Software für den Endnutzer bedient werden kann.
Über das Menü kann kann die Gameselection augerufen werden, bei welcher die verfügbaren Spiele ausgewählt werden können.
Wird ein Spiel ausgewählt, gibt es ein weiteres Menü über dass Spielspezifische Parameter einstellbar sind. 
Die Spiele sind als eigene Pakete eingegliedert und behinhalten, je nach Spiel ein Spielbrett und Möglichkeiten mittels Touch-Berührungen
Züge auszuführen. Durch diese Untergliederung in weitere Pakete kann die Software leicht um weitere Spiele erweitert werden.

\vspace{1em}
\begin{minipage}{\linewidth}
	\centering
	\includegraphics[width=0.9\linewidth]{pics/ReversiXTGUIClassDiagramm.png}
	\captionof{figure}[ReversiXTGUI]{ Das UML Klassendiagramm der ReversiXT GUI }
	\label{fig:ReversiXTGUIClassDiagram}
\end{minipage}

\subsubsection{Reverse Proxy}

\subsection{KI Client}
Der KI Client beinhaltet die Logik der Künstlichen Intelligenz der Applikation. Wird ein Spiel gegen einen KI Client gestartet, so wird dieser gestartet und agiert als 
Gegenspieler zum User. Der User kann auch ein Spiel bei welchem zwei KI's gegeneinander spielen starten, wodurch zwei KI Clients gestartet werden.

\subsubsection{Softwareaufbau des KI Clients}
Der KI Client besteht aus drei Paketen, der KI-Logik, der Spiellogik und der Serververbindungslogik, siehe \ref{fig:KIClientClassDiagram}. 
Die Spilelogik, beeinhaltet ein Momentanzustand des Spielbrettes, sowie Möglichkeiten dieses nach belieben zu modifizieren und ist in Kapietel 
\ref{chap:Spiellogik} erklärt. Das KI-Paket benutzt diese Logik um 
Züge auf auszuwerten und den bestmöglichen Zug zu wählen. Um verschiedene KI-Algorithmen verwenden zu können wird ein Interface bereit gestellt, wird die Applikation 
um einen weiteren KI-Algorithmus erweitert, so muss dieser das Interface verwenden. Wird eine Auswertung des Nächsten besten Zuges abeschlossen, so sendet das KI-Paket sein
Ergebnis an die das Server-Paket. Dieses ist verantwortlich für das Versenden und kodieren der Nachrichten. Nachrichten werden vom KI-Client zum Gameserver geschickt und müssen
dabei das Protokoll von diesem einhalten.

\vspace{1em}
\begin{minipage}{\linewidth}
	\centering
	\includegraphics[width=0.9\linewidth]{pics/GameClientClassDiagram.png}
	\captionof{figure}[KIClient]{ Das UML Klassendiagramm des KI Clients }
	\label{fig:KIClientClassDiagram}
\end{minipage}


\pagebreak
\section{Hardware}
Dieses Kapitel handelt von der verwendeten Hardware für welche die Software erstellt worden ist. 
Die Ursprüngliche Idee des Projektes ist, einen Vorzeigegegenstand für Messen, wie den Tag der Informatik, zu haben.
Um dies zu realisieren, besteht das Projekt aus einen Touchscreen, welcher in einem Tischförmigen Gerüst verankert 
und mit einen Raspberry Pi verbunden ist.

\subsection{Raspberry Pi}
Die Software wird auf einen Raspberry Pi Model 3 ausgeführt. Bei dem Betriebssystemen handelt es sich hierbei um 
Raspbian, ein auf Debian Basiertes Linux Betriebssystem. \cite{RaspberryPi} Da die Software auch mit anderen Betriebssystem
kompatibel ist, könnte man den Pi durch ein beliebig anderes Gerät ersetzen, dass entweder Linux oder Windows unterstüzt.
Der Raspberry Pi ist jedoch anhand seiner Kosten und Größe perfekt für das Projekt, da er in das Gerüst passt.

\subsection{Touch Monitor}
Die Bedienung der Benutzerobefläche erfolgt hauptsächlich mit dem Touchscreen, jedoch kann die Software auch auf normalen Monitoren ausgeführt werden.
Der Touch Monitor ist ...% todo Tochscreengröße einfügen und Marke
Eingaben können entweder durch Berührung des Touchscreens mit dem Finger, oder mit einer klassischen Computermaus ausgeführt werden.
Für Texteingabe, kann entweder das integrierte Softwarekeyboard, oder eine normale Tastatur verwendet werden.

%todo bild der Hardware


\pagebreak
\section{Implementierung}

\subsection{Programmiersprachen und Frameworks}
Da die Software aus Mehreren individuellen Applikationen besteht, werden auch verschiedenste Programmiersprachen und Frameworks for diese Verwendet.
Im Folgenden werden alle Programmiersprachen und Frameworks die Verwendung finden erklärt.

\subsubsection{React.js}
Um die Software so kompatibel wie möglich zu gestalten, basiert die Software auf Webtechnologie. Dies bedeutet, jedes Endgerät, welches einen neuren 
Internetbrowser unterstüzt kann die Software zu einem Teil ausführen. Da eine Webseite in reinen Javascript, HTML und CSS zu schreiben viel Aufwand benötigt, 
wird ein Framework wie React.js verwendet. Bei React handelt es sich um eine Javascript Library zum erstellen von Benutzeroberflächen.
Die Vorteile von React zu reinem Javascript sind:
\begin{itemize}
    \item Einfach Dynamische Websiten zu entwickeln
    \item Wiederverwertbare Komponenten
    \item Methoden und Markup gehören zusammen
    \item Funktionale Programmierung mit pure functions
\end{itemize}
Vorallem das erleichterte Erstellen von Dynamischen Websiten ist Vorteilhaft, da die Benutzerobefläche wärend ein Spiel gespielt wird, dynamisch gehalten werden muss.
Im klassischen Javascript muss das Document Object Model (DOM), mühsam verändert werden. Diese DOM-Manipulationen sind sehr Fehleranfällig und können Memoryleaks
verursachen. React sitzt hingegen zwischen Komponenten und dem DOM und übernimmt die komplette DOM-Manipulation.
Das Verwenden von Komponenten ist auch Vorteilhaft, da zum Beispiel einzelne Spielfiguren als Komponenten festgelegt werden können 
und so Code mehrfach eingespart werden kann. \cite{React.js}.

\subsubsection{Node.js}
Der Gameserver, der Gameclient und die Serverlogik der GUI, sind mit Node.js implementiert. 
Node.js ist eine asynchrone ereignisgesteueretes Javascript-Laufzeitumgebung, welche Javascript-Code außerhalb des Webbrowsers ausführen kann.
Node.js wird hauptsächlich zur Programmierung von Netzwerkanwendungen, wie Webservern verwendet. Da einen Großteil der Anwendung die
Kommunikation der einzelnen Komponenten ausmacht, ist Node.js perfekt für dieses Szenario. 
Ein weiterer Vorteil von Node ist, dass Node.js unabhängig vom Betriebssystem ist. Dies bedeutet, kann die V8 Javascript-Laufzeitumgebung
auf dem Betriebssystem ausgeführt werden, so kann man auch die Node-Applikation dort zum laufen bringen.
\cite{Node}

\subsubsection{Typescript}
Node.js wir eigentlich in reinem Javascript geschrieben, jedoch leidet Javascript daraunter, dass es keine Typisierung hat und sich somit leicht Fehler 
einschleichen können. Typescript ist eine auf Javascript basierende Programmiersprache, welche statische Typisierung unterstützt. 
Ähnlich wie bei C und C++ ist valider Javascript-Code auch valider Typescript code wodurch Typescript eine Obermenge von Javascript ist.
Nachdem Typescript-Code geschrieben wird, kann er in reines Javascript kompiliert werden. 
Der Gameserver und der Gameclient sind in Typescript geschrieben um Fehler durch die Typsicherheit zu verhindern. 

\subsubsection{C/C++ Node Addons}
Da Javascript eine im Vergleich sehr langsame Programmiersprache ist, wäre es sinnvoll den künstliche Intelligenz Teil der Applikation, welcher für die 
Berechnung der nächsten Züge verantwortlich ist, in einer Performanteren Sprache welche auch Memorymanagement bietet, wie C oder C++ zu schreiben. 
Node.js bietet hierfür die Möglichkeit an
Module mittles C++ zu implementieren und von deren Geschwindigkeit zu profitieren \cite{NodeC++Performance}. Ein C++ Node Addon wird zu Maschinencode kompiliert, welcher von der
normalen Node-Anwendung, über eine Javascript-Funktion ausgeführt werden kann. Der Vorteil hierbei zum reinen C++ ist, dass der Code in der V8 
Laufzeitumgebung ausgeführt wird, was ihn Plattformunabhängig macht. \cite{C++Node}

%todo grafik die Laufzeitverbesserung zeigt

\subsection{Kommunikation}
in \ref{chap:Ablauf} ist bereits auf die Kommunikationsstruktur geschildert, im folgenden wird die genauere Implementierung beschrieben

\subsubsection{Verwendete Technolgie}
Da der Gameserver, der Gameclient und die GUI eigenständige Prozesse sind, die nach bedarf gestartet und getstoppt werden, brauchen sie eine 
Möglichkeit Nachrichten auszutauschen. Dafür werden die von Node.js bereitgestellten TCP-Sockets verwendet. Der Vorteil von TCP-Sockets 
gegenüber zum Beispiel UDP, ist dass aufrechterhalten einer Verbindung. So muss sich der Client oder die GUI nur einmal zum Server verbinden 
und kann im Laufe eines gesamten Spieles diese Verbindung nutzen. Ist ein Spiel zu Ende, kann die TCP-Verbindung gekappt und die 
Sockets gescholossen werden. Da die GUI in zwei Softwarepakete getrennt ist siehe \ref{fig:ReversiXTGUIClassDiagram}, der React.js Webapp und einem Node.js Server, verwenden 
diese Pakete auch eine eigen Netzwerkverbindung. Hierfür wird die Javascript library Socket.io verwendet, welche auf basis von Websockets basiert.

\subsubsection{Kommunikationsprotokoll des Gameservers}
Im Abschnitt \ref{chap:Ablauf} wird eine Kommunikation zwischen dem Gameserver und seinen Clients beschrieben, bei welcher der 
Server nur Anfragen eines festgelegten Protokolles Akzeptiert. Das Protokoll für den Server orientiert sich stark nach dem Protokoll
des Reversi-Gameservers und ist nur in einigen stellen erweitert worden.
%Nachdem ein Spiel gestartet wurde, sendet der Gameserver den Initialzustand des Spielbrettes an die Clients und
%eine Zugaufforderung an den Spieler der den ersten Zug ziehen darf. Der Client sendet daraufhin eine Zugantwort.
%Dieser Zug wird auf Richtigkeit überprüft und falls er in Ordnung ist auf das Spielfeld angewendet. 
%Der Server schickt je nachdem welcher Client an der Reihe ist Zugaufforderungen und die Clients erwiedern diese mit Zugantworten.
%Das Spiel endet, falls ein Client einen ungültigen Zug schickt, oder ein Endzustand des Spieles (Gewinn, Verlust, Unentschieden) 
%erreicht wird. Nachdem das Spiel endet, beendet der Server die Kommunikation mit den Clients mit einer Nachricht und Beendet sich selbst
%
\begin{table}[H]
    \centering
    \begin{tabular} {|c|c|c|}
        \hline
        Typ (8-Bit-Integer) & Länge der Nachricht n (32-Bit-Integer) & Nachricht (n Bytes) \\
        \hline
    \end{tabular}
	\captionof{figure}{Der Aufbau einer Nachricht}
    \label{fig:Nachritenaufbau}
\end{table}

in Abbildung \ref{fig:Nachritenaufbau} ist der Aufbau einer Nachricht dargestellt. Es gibt ingesamt neun verschiedene Nachrichtentypen, dazu gehören 
neues Spiel starten, Zugaufforderung, Spiel Ende und viele weitere. Die meisten Nachrichten von einen gewissen Typ haben immer die selbe Länge, es gibt jedoch
Ausnahmen, wie die Spielbrettnachricht, bei welcher ein 10x10 oder 8x8 Damenspielbrett gewählt werden kann. Die Nachricht selber entspricht dann den eigentlichen 
Daten, die übertragen werden, wie zum Beispiel, den Feldern welche besetzt sind beim Spielfeld, oder das Ursprungsfeld und das Zielfeld bei einer Zugantwort. 

\subsection{GUI Dame Erweiterung}
Da die Anwendung vor der Erweiterung um das Spiel Dame nur Reversi als Spiel hatte, musste ein Extra Menü implementiert werden, über welches man Dame auswählen kann.
Des weiterem wird ein weiteres Menü benötigt, bei welchem der Algorithmus und die Spielbrettgröße gewählt werden kann, siehe Abbildung \ref{fig:Auswahlmenue}.
Dieses Menü zeigt eine Auswahl für Spielfeldgrößen, 
%todo ausfuehrlcher

\vspace{1em}
\begin{minipage}{\linewidth}
	\centering
	\includegraphics[width=0.7\linewidth]{pics/AlgorithmeninderApplikation.png}
	\captionof{figure}[Auswahlmenü]{ Das Auswahlmenü für ein Damespiel }
	\label{fig:Auswahlmenue}
\end{minipage}

\subsection{Verwendete KI-Algorithmen}
In Kapitel \ref{chap:Grundlagen} wurden einige künstliche Intelligenz Algorithemen beschrieben, welche auch im Dame-Spiel der Applikation implementiert sind.
Die Applikation bietet vier Algorithemen zur Auswahl an, gegen die der Spieler antreten kann, siehe \ref{fig:Auswahlmenue}. 
Diese Algorithemen sind: Minimax, Alpha-Beta-Pruning, Zugsortierung und MCTS.
Aus Abbildung \ref{fig:KIClientClassDiagram}
kann man entnehmen, dass es zwei Hauptalgorithmen in dem Paket CheckersAiLogik gibt, Minimax und MCTS. Diese Aufteilung kann vorgenommen werden,
da Alpha-Beta-Pruning und Zugsortierung auf Minimax basieren.


\subsubsection{Minimax}
Der Minimax-Algorithmus wird im wie in \ref{chap:Minimax} erklärt, durch einen Baum repräsentiert. Dadurch, dass jeder Knoten einen Spielbrettzustand
repräsentiert und durch den Algorithmus einen Wert zugewiesen bekommen, hat die Applikation eine MinimaxNode-Klasse.
Diese Klasse ist hauptsächlich für das Speichern der Informationen zuständig, welche dann einfach von der eigentlichen Minimax-Klasse zur Berechnung
verwendet werden können. Damit ein vollständiger Baum aufgebaut werden kann, enthält die Minimax-Node-Klasse ein Array mit Zeigern auf allen 
Kindknoten, was eine Vorwärtstraversierung des Baumes erlaubt. 
Um eine Berechnung zu starten, braucht die start-Funktion ein Spielbrett, den Spieler der an der Reihe ist, eine
Tiefe bis zu der berechnet werden soll und oder ein Zeitlimit. Ist ein Zeitlimit angegeben, so benutzt die start-Funktion Iterative Deepening, siehe
\ref{chap:IterativeDeepening}, dadurch kann die Berechnung so gut wie Möglich die komplette Zeit ausnutzen. Ist kein Zeitlimit angegeben, 
so wird spezifizierte Tiefe dem Minimax übergeben und dieser rechnet so lang bis er diese erreicht. Wird Minimax mit Iterative Deepening gestartet, so
wird ein Zeitwert zum Anfang gespeichert. Dieser Zeitwert wird wärend der Berechnung von Minimax mit dem momentanen Zeitwert verglichen, ist die 
Differenz größer oder gleich des Zeitslimits, wird ein mittels einer C++ Exception die Momentane Berechnung abgebrochen und nachdem der zurückkehren in 
den Catch-Block der letzte komplett berechnete Zug verwendet. 
Wird Minimax gestartet, holt sich der Algorithmus mittels der Verwendung der Funktion AddAllChildren, alle Kinderspielbretter. Diese 
definieren sich aus allen möglichen Zügen die von einer Stellung aus möglich sind, siehe Kapitel \ref{chap:Spiellogik}. Der Vorgang wird 
rekursiv wiederholt, bis die geforderte Tiefe erreicht wird und die Bewertungsfunktion aufgerufen wird. Die Ergenisse werden dann rekursiv rückwärts in 
den MinimaxNodes des Baumes eingetragen.

Um mit Minimax den besten Zug berechnen zu können wird eine gute Bewertungsfunktion 
benötigt, die verwendete hat folgende Eigenschaften:
\begin{itemize}
    \item Normale Figur: +1 für Weiß und -1 für Schwarz
    \item Dame: +3 für Weiß und -3 für Schwarz
    \item Spielende: +40 für Weiß und -40 für Schwarz
\end{itemize} 
Erweiterungsmöglichkeiten wären, Bedrohte Figuren, sowie Figuren in der Spielmitte und in den Grundreihen mit in die Berechnung miteinfließen zu lassen.
Man muss ich aber immer vor Augen halten, dass eine komplexere Bewertungsfunktion mehr Berechnungszeit benötigt, aber bessere Ergebnisse liefern könnte.

\subsubsection{Minimax Erweiterungen Alpha-Beta-Pruning und Zugsortierung}
Dadurch, dass Alpha-Beta-Pruning und Zugsortierung Erweiterungen von Minimax sind, können diese beiden Algorithmen mit in die Minimax-Klasse gepackt
und über Übergabeparameter des Konstruktors an- und ausgeschalten werden. Dadurch hat man weniger Code, kann aber trozdem noch 
verschieden Algorithmen mit unterschieder Stärke im Menü wählen. Da Minimax rekursiv implementiert ist, kann Alpha-Beta-Pruning sehr elegent mit wenigen 
Zeilen Code geschrieben werden. In Listing \ref{lst:AlphaBetaCode} sieht man die Einfache Implementierung des Alpha-Beta-Prunings im Maximierer Teil 
von Minimax. Zuerst wird überprüft ob es sich um reinen Minimax handlet, oder um Alpha-Beta oder Zugsortierung. Danach wird der Alpha Wert mit dem
Maximum aus dem alten Alpha-Wert und dem durch die Bewertungsfunktion berechneten eval Wert berechnet. Das Codestück im Minimierer hätte an dieser Stelle 
ein Min anstelle des Max. Das Pruning finden für den fall dass Beta kleiner gleich Alpha ist statt, dadurch wird der Teil dieses Zweiges nicht weiter untersucht.
\vspace{1em}
\lstinputlisting[caption=Code der Implementierung des Alpha-Beta-Prunings, label=lst:AlphaBetaCode,basicstyle=\ttfamily\scriptsize]{code/Alpha-Beta.txt}

Um eine Zugsortierung im Minimax zu erreichen, wird im Teil von AddAllChildren des Minimax die Bewertungsfunktion der Zugsortierung aufgerufen.
Dadurch dass es sich um Array von Zeigern handelt, welche auf die Kindknoten zeigen muss nur dieses Array unter Berücksichtigung der Bewertungsfunktion
sortiert werden, was durch die Hilfe der Standartbibliotek sehr einfach zu implementieren ist.  
% todo Bewertungsfunktion erklären%
Die Verbessung wird nun durch das Aufrufen des Alpha-Beta-Prunings erreicht, siehe Kapitel \ref{chap:Zugsortierung}.



\subsubsection{MCTS}
Der MCTS-Algorithmus ist ein auf Simulation basierender Algorithmus, er ist in Kapietel \ref{chap:MCTS} erklärt worden.
Ähnlich wie beim Minimax-Algorithmus verwendet der MCTS-Algorithmus auch einen Baum mit Knoten welche den Spielbrettzustand halten.
Der Unterschied liegt hierbei bei den Informationen, die ein Konten hält. Eine MCTSnode hält die Anzahl an verlorenen und gewonnen Spielen,
die aus der Simulation hervorgehen. Jeder Knoten hat ein Array mit Zeigern auf seine Kinderknoten, dadruch entseht die bekannte Baumstruktur.
Im Gegensatz zur MinimaxNode, werden nicht bei einer Iteration alle Kinder eines Knoten mit AddAllChildren hinzugefügt, sondern mittels des 
UCTS-Ergbnisse, wird der beste Kandidat gewählt. Für den Parameter $c$ wird zwei gewählt, da in vielen gelungen anderen Spielen die Zahl zwei 
zu guten erfolgen geführt hat. Da der MCTS nicht unendlich lange simulieren soll, braucht er ein Zeitlimit oder eine Anzahl an Simulationen
nach welchen er die Simulation abbrechen und das Ergebnis zurückliefern soll. Die Zeitüberprüfung wird ähnlich wie bei Minimax gehalten,
ist der Zeitpunkt zum abbruch gekommen wird eine C++ Exception geworfen, was zum Stop der Simulation führt. 
Um die Simulation Ausführen zu können werden die Methoden, welche in \ref{chap:Spiellogik} beschrieben sind verwendet.
Das Ergebnis des MCTS ist der Kindknoten, welcher den besten Wert nach den Simulationen hat, dieser wird auch vom Algorihtmus zurückgegeben.

\subsection{Dame Spiellogik}
\label{chap:Moves}
Im Kapitel \ref{chap:Moves} ist bereits der Aufbau der Spiellogik erklärt. Die Besonderheit bei der Implementierung ist, dass sich der Gameserver und 
der Gameclient in der Implementierung des MoveAppliers und CheckersRules unterscheiden. Der Grundliegende Unterschied ist, dass 
beim Berechenen eines neuen Zuges durch den Gameclient sehr oft die Methode "`move"' aufgerufen werden muss. Da diese Züge nicht von 
anden Komponenten der Anwendung kommen, müssen sie nicht auf Korrektheit überprüft werden. Eine der Hautpaufgaben des Gameservers ist die 
Überprüfung und Validierung der eingehenden Züge, diese Validierung kostet aber auch Zeit welche die Berechnung im Gameclient verlangsamen würden 
und daher im Gameclient bei der Berechnung weggelassen wird.


\section{Testing und Simulation}
\subsection{Simulation}
Da die Software mehrere KI-Algorithemen beeinhaltet und diese unterschiedlich stark sind, werden Simulationen verwendet um die erwartete Stärke der Algorithemen
zu bestätigen. Im laufe der Simulationen spielt jede KI gegen die anderen KI's. Um die Ergebnisse in Form von Stärke festzuhalten, bekommt jeder Algorithmus ahnhand
seiner Performance eine ELO-Zahl zugewisen.

\subsubsection{ELO}
Das Elo System ist eine Kennzahl für die relative Spielstärke die ein Spieler in einem Nullsummenspiel hat. ELO wird hauptsächlich in Schach und Go verwendet,
findet aber auch immer mehr Anwendung in anderen Sportarten wie Tischtennis oder auch in Computerspielen. Bei der Berechnung von Elo wird ein Spiel von zwei 
Spielern untersucht, dabei ist die Elozahl der Spieler die erwartete Spielstärke. Demnach wird von einem Spieler mit höherer Elo-Zahl als sein Gegener
vielmehr ein Gewinn erwartet. Dementsprechend verhält sich auch die Eloänderung nach dem Spiel. Verliert ein Spieler A mit einer höheren Elo gegen einen
Spieler B mit gerigerer Elo, so verliert A viel Punkte und B bekommt viel Punkte dazu. Gewinnt jeoch A, so bekommt er nur sehr wenig Punkte hinzu und B verliert 
nur sehr wenig. \\ Die in der Applikation verwendete Elo Zahl wird wie folgt berechnet:

\begin{align}
    R_1 = 10^{\frac{E_1}{400}} \\ 
    R_2 = 10^{\frac{E_2}{400}} \\
    F_1 = \frac{R_1}{R_1+R_2} \\ 
    F_2 = \frac{R_2}{R_1+R_2} \\
    E_{neu1} = E_1 + k \cdot (S_1 - F_1) \\ 
    E_{neu2} = E_2 + k \cdot (S_2 - F_1)  
\end{align}

Dabei entspricht:
\begin{conditions}
    E     &  Elo Zahl \\   
    F     &  Warscheinlichkeit das der Spieler gewinnt \\
    k     &  K-Faktor \\
    S     &  Sieg oder Niederlage (1 oder 0)
\end{conditions}

Der K-Faktor gibt eine Aussage darüber wie stark sich ein Spiel auf die Eloänderung auswirkt. In der Applikation wurde ein K-Faktor von 30 verwendet, da 
dieser auch in Turnierschach für Spieler unter 2100 Elo verwendet wird.

\subsubsection{Aufbau der Simulationsoftware}
Für die Simulation wurde eine Eigene Software geschrieben, welche unabhängig von der ReversiXT GUI arbeitet. 
Es werden lediglich der Gameserver und der Gameclient benötigt. Um eine Simulation zu starten muss eine Anzahl durchläufen angeben werden, je nach dem
wie viele Spiele ausgeführt werden sollen. Startet man ein Spiel wird der Gameserver und zwei Gameclients mit zufällig sich unterscheidenden Algorithmen gestartet.
Wird ein Spiel abgeschlossen wird das Ergbnis in einer Datei gespeichert und die Elo aktualisiert.

\subsubsection{Interpretation der Ergebnisse}
Die vier implementierten Algorithemen der Applikation sind:
\begin{itemize}
    \item MCTS
    \item Minimax 
    \item Alpha-Beta Pruning
    \item Alpha-Beta mit Zugsortierung 
\end{itemize}
Man könnte erwarten dass Alpha-Beta mit Zugsortierung die höchste Elozahl, Alpha-Beta Pruning die zweithöchste, Minimax die dritthöchste und MCTS die geringste Elo-
Zahl erhalten würden. Diese Annahme lässt sich durch die Simulation bestätigen.

%todo -> Bild mit werten einfügen

\subsection{Testing}
Um die Stabilität der Software, zu verbessern, sind Tests für die anfälligen Softwarekomponenten erstellt worden, durch welche gröbere Fehler vermieden werden können.
Die GUI ist weniger Fehleranfällig, da in ihr lediglich Informationen dargestellt werden. Die Kritischen Softwareteile sind der Gameserver und der Gameclient, 
in welchen das Dame Spiel ausgeführt und neue Züge berechnet werden. Ein Fehler dort wäre verherend, da die Software nicht in einen Korrekten Zustand zurückgeführt
werden kann.

\subsubsection{Gameserver}
Im Gameserver ist der Zustand des Spieles und die Überprüfung der gespielten Züge. Diese Überprüfung darf auch in Spezialfällen nicht fehlschlagen, so muss z.B. ein
Spieler in Schlagzwang immer das Schlagen einer Figur ausführen, oder ein Spieler kann nicht noch einmal schlagen, wenn er mit einem Zug bei dem geschlagen worden ist,
eine Dame erhält. Um dies zu überprüfen, gibt es Tests, die Spezielle Spielsituationen als Eingabe bekommen und prüfen ob die Software den Zug als Gültig/Ungültig
anerkennt. Die Situationen sind entweder Spezialfälle die aus den Damenregeln hervorgehen, oder Standartstellungen bei denen eigentlich kein Fehler auftreten dürfte.
Die Testsoftware verwendet das Testing Framework Chai, welches zum testen von Node Anwendungen verwendet wird. 
Dabei werden mehrere Testfällte beschrieben, diese werden dann über asserts oder expects von Chai ausgewertet. Diese Auswertung verläuft automatisiert,
dass heißt wird die Software neu von Typescript in Javascript kompiliert, werden auch die Testfälle ausgeführt.

\subsubsection{Gameclient}
Das Testen des Gameclients ist sehr wichtig, da kleinste Fehler zu einer enormen Verschlechterung der Performance des KI-Algorithmuses führen.
Um den Gameclient zu testen, verwendet man ähnlich wie beim Gameserver Szenarien, in denen es schwirig ist den besten Zug zu finden, oder 
in sehr einfachen Situationen, bei dem der KI-Algorithmus nicht sehr lagen brauchen sollte um den Perfekten Zug ausfindig zu machen.
Da die KI-Logik des Gameclients in C++ implementiert ist, ist die Testsoftware manuel geschrieben. 
Was bedeutet, dass es handelt sich um eine unabhängige Applikation, welche einen Gameclient startet und diesem mit Spielsituation und Zugaufforderung füttert.
Der Client gibt der Testsoftware, dann den nächsten Zug zurück. Das Ergbnis wird mit dem gewünschten Ergbnis verglichen, um zu bestimmen ob ein Fehler 
oder eine Korrekte Ausgabe getätigt werden muss. In diesem Prinzip startet die Testsoftware den Gameclient mehrfach für verschiedenste Testfällte und
wertet diese Ergbnisse aus.


\pagebreak
% ----------------------------------------------------------------------------------
% Kapitel: Fazit und Ausblick
% ----------------------------------------------------------------------------------
\section{Fazit und Ausblick}



\pagebreak
% ----------------------------------------------------------------------------------
% Kleine Einführung in LaTeX-Elemente
% ----------------------------------------------------------------------------------
%\section{\LaTeX-Elemente}
%Dieser Abschnitt beinhaltet lediglich einige Informationen über \LaTeX-Distributionen, Editoren und \LaTeX-Elemente, die Ihnen beim Einstieg in das \LaTeX-Textsatzsystem helfen sollen.
%
%\subsection{\LaTeX-Distributionen nach Betriebssystemen}
%
%\subsubsection{\LaTeX-Distributionen}
%Folgende Haupt-\LaTeX-Distributionen stehen Ihnen zur Verfügung:
%\begin{itemize}
%  \item Windows:\quad \texttt{MiKTeX}\quad Webseite:\quad\url{http://www.miktex.org}
%  \item Linux/Unix:\quad \texttt{TeX Live}\quad Webseite:\quad\url{http://tug.org/texlive/}
%  \item Mac OS:\quad \texttt{MacTeX}\quad Webseite:\quad\url{http://www.tug.org/mactex/}
%\end{itemize}
%
%\subsubsection{\LaTeX-Editoren}
%Auf folgenden Webseiten können Sie einige hilfreiche \LaTeX-Editoren finden:
%\begin{itemize}
%  \item Windows/Linux/Mac OS: \url{http://www.xm1math.net/texmaker/}
%  \item Windiws: \url{http://www.texniccenter.org/}
%  \item Mac OS: \url{http://pages.uoregon.edu/koch/texshop/}
%\end{itemize}
%
%Falls bei den oben genannten Editoren kein passender vorhanden war, findet sich auf Wikipedia eine Zusammenstellung vieler weiterer \LaTeX-Editoren:\\[1em]
%\hspace*{3cm}\url{https://en.wikipedia.org/wiki/Comparison_of_TeX_editors}
%
%
%\subsection{Bilder}
%Zum Einfügen eines Bildes, siehe Abbildung \ref{fig:reversi01}, werden die \texttt{minipage}-Umgebung und der Befehl \texttt{$\backslash$includegraphics} genutzt, da die Bilder so gut positioniert und einfach integriert und skaliert werden können.
%
%\vspace{1em}
%\begin{minipage}{\linewidth}
%	\centering
%	\includegraphics[width=0.5\linewidth]{pics/gamefield01.png}
%	\captionof{figure}[Spielfeld 01]{Unbespieltes Spielfeld\footnotemark }
%	\label{fig:reversi01}
%\end{minipage}
%\footnotetext{Diesem Spielfeld wurden noch keine Spieler zugewiesen (daher die dunklen Spielsteine)}
%
%Nachdem das Spielt gestartet wurde und beide Spielphasen durchlaufen wurden, siegt schließlich der Spieler mit der Farbe rot.
%
%\vspace{1em}
%\begin{minipage}{\linewidth}
%	\centering
%	\includegraphics[width=0.5\linewidth]{pics/gamefield02.png}
%	\captionof{figure}[Spielfeld 02]{Finales Spielfeld\footnotemark }
%	\label{fig:reversi2}
%\end{minipage}
%\footnotetext{Das Spielfeld nach der Zug- und Bombenphase. Spieler rot gewinnt eindeutig.}
%
%\subsection{Tabellen}
%In diesem Abschnitt wird eine Tabelle (siehe Tabelle \ref{tab:beispiel}) dargestellt.
%
%\vspace{1em}
%\begin{table}[!h]
%	\centering
%	\begin{tabular}{|l|l|l|}
%		\hline
%		\textbf{Name} & \textbf{Name} & \textbf{Name}\\
%		\hline
%		1 & 2 & 3\\
%		\hline
%		4 & 5 & 6\\
%		\hline
%		7 & 8 & 9\\
%		\hline
%	\end{tabular}
%	\caption{Beispieltabelle}
%	\label{tab:beispiel}
%\end{table}
%
%
%\subsection{Auflistung}
%Für Auflistungen wird die \texttt{enumerate}- oder \texttt{itemize}-Umgebung genutzt.
%
%\begin{itemize}
%	\item Nur
%	\item ein
%	\item Beispiel.
%\end{itemize}
%
%\subsection{Listings}
%Zuletzt sehen Sie in Listing \ref{lst:maxTeilsumZweiD} ein Beispiel für das Einbinden von Quellcode mit Syntax-Highlighting.
%
%\vspace{1em}
%\lstinputlisting[caption=Brute Force-Ansatz für das MaxTeilsum2D-Problem, label=lst:maxTeilsumZweiD,basicstyle=\ttfamily\scriptsize]{code/maxTeilsum2DBruteForce.txt}
%
%\subsection{Selbstgestaltete Abbildungen}
%Mithilfe des Paketes \texttt{tikz} können sehr schöne Abbildungen (z.\,B.\ Automaten, Graphen etc.) direkt in \LaTeX generiert werden. Viele Beispiele dazu finden Sie auf folgender Webseite:\\[1em]
%\hspace*{3cm}\url{http://www.texample.net/tikz/}.
%
%\subsection{Tipps}
%Die Literaturreferenzen (Bücher, Paper und Journals) und Internetquellen (Webseiten, Blogs etc.) befinden sich in der Datei \textit{literatur.bib}. Eine Buch- und eine Online-Quelle sind beispielhaft eingefügt. [Vgl.\ \cite{buch}, \cite{mathcomm}]
%
%Literatur und Quellen werden in zwei getrennte Verzeichnisse aufgeteilt. Als Unterscheidungsmerkmal dient bei den Quellen der Zusatz: \texttt{keywords = \{online\}}.

\pagebreak

% ----------------------------------------------------------------------------------------------------------
% Filter fuer Literatur und Quellen definieren
% ----------------------------------------------------------------------------------------------------------

\defbibheading{Literatur}{\section*{Literaturverzeichnis}} 
\defbibheading{Quellen}{\section*{Quellenverzeichnis}} 
  
\defbibfilter{Literatur}{\not\keyword{online}} 
\defbibfilter{Quellen}{\keyword{online}} 


 ----------------------------------------------------------------------------------------------------------
 Literatur
 ----------------------------------------------------------------------------------------------------------
\lhead{} 
\rhead{Literaturverzeichnis} 

\printbibliography[heading=Literatur,filter=Literatur] 

\pagebreak


% ---------------------------------------------------------------------------------------------------------- 
% Quellen 
% ---------------------------------------------------------------------------------------------------------- 
\lhead{} 
\rhead{Quellenverzeichnis} 

\printbibliography[title = {Quellenverzeichnis}, heading=Quellen,filter=Quellen] 

\pagebreak 

% ----------------------------------------------------------------------------------------------------------
% Anhang
% ----------------------------------------------------------------------------------------------------------
\pagenumbering{Roman}
\setcounter{page}{1}
\lhead{Anhang \thesection}

\begin{appendix}
\section*{Anhang}
\phantomsection
\addcontentsline{toc}{section}{Anhang}
\addtocontents{toc}{\vspace{-0.5em}}

\section{Anforderungsanalyse}
\label{apx:Anforderungsanalyse}
In diesem Kapitel wird zuerst ein beispielhaftes Szenario gezeigt, bei welchem die Software eingesetzt werden soll.
Anschließend werden anhand dieser Szenarios Anforderungen definiert, die von der Software erfüllt werden müssen.
\subsection{Anwendungsszenario}
Will ein Benutzer gegen den KI Client spielen, kann er über das Menü auf dem Touch Monitor den KI Algorithmus auswählen und diesen herausfordern.
Dazu kann er, falls er mit seinem Smarthphone spielen will einen QR-Code auf dem Monitor aufrufen, diesen einscannen und dann das Spiel beginnen.
Andernfalls kann der Benutzer auch direkt am Monitor das Spiel starten und auf diesen mittels Touch Züge ausführen.
Die KI Algorithmen welche auf dem KI Client implementiert sind haben eine ELO Zahl hinterlegt, welche Auskunft über die Spielstärke gibt.
Fühlt sich der Benutzer also über oder unterfordert, kann er den geeigneten Gegner auswählen. Gibt es mehrere Benutzer welche 
gegen einander spielen wollen, haben diese die Möglichkeit entweder abwechselnd auf dem Monitor, oder beide mittels QR-Code über ihr Smartphone ein Spiel zu starten.
Will ein Benutzer zwei KI's beim Spielen beobachten, so kann er diese am Monitor auswählen und diese gegeneinander Antreten lassen.
\subsection{Anforderungen an die Software}
Aus dem oben beschriebenen Anwendungsszenario lassen sich konkrete Anforderungen ableiten, die für die Software von Relevanz sind.
Hierbei wird zwischen funktionalen und nicht funktionalen Anforderungen unterschieden \cite{RequirementEngenieering}.
\subsubsection{Funktionale Anforderungen}
\begin{itemize}
    \item \textbf{/F10/} \textit{Menü zum Auswählen des Spieles:} Die Grafische Oberfläche soll dem Benutzer die Möglichkeit geben, das ein Spiel auszuwählen.
        Dazu soll es ein Menü geben welches die möglichen Spiele, wie z.B. Reversi oder Dame zur Auswahl stellt.
    \item \textbf{/F11/} \textit{Auswählbares Menü für verschiedene KI Algorithmen:} Die Applikation muss ein leicht bedienbares Grafisches Interface bieten,
        bei welchem verschiedene künstliche Intelligenz Algorithmen ausgewählt und herausgefordert und werden können.
    \item \textbf{/F12/} \textit{QR-Code fürs verbinden mit dem Smartphone:} Um einen unkomplizierten Verbindungsaufbau vom Raspberry mit dem Smarthphone zu 
        gewährleisten, soll es eine Menü-Option geben, bei der ein QR-Code angezeigt wird. Nach dem Scannen des QR-Codes soll eine Verbindung
        aufgebaut werden, welche bis zum Beenden bestehen bleibt.
    \item \textbf{/F13/} \textit{Varaible Zeiteinstellung im Menü:} Dem Benutzer soll es möglich sein über das Menü eine Zeit einstellen zu können, 
        welche jeder Spieler im Spiel zur Verfügung für seine Züge hat. Als Spieler können entweder Benutzer oder KI Clients agieren.
    \item \textbf{/F20/} \textit{Benutzer soll Züge ausführen können:} Der Benutzer soll in der Lage sein, Züge gegen die KI spielen zu können. Dazu soll er entweder
        direkt über den Touch-Monitor oder über das Smartphone eine Eingabemöglichkeit haben. Diese soll den Momentanzustand des Spielbrettes zeigen,
        wodurch der Benutzer eine Entscheidung für seinen nächsten Zug treffen und diese über eine Touch-Berührung ausführen kann.
    \item \textbf{/F21/} \textit{Benutzer sollen gegen andere Benutzer spielen können:} Für mehrere Benutzer soll es möglich sein, gegeneinander spielen zu können. Dazu sollen sie 
        entweder den Touch-Monitor verwenden, indem sie abwechselnd Züge ausführen, oder beide jeweils ein Smarthphone.
    \item \textbf{/F22/} \textit{Der Benutzer kann KI's gegeneinander spielen lassen:} Der Benutzer soll in der Lage sein zwei KI-Algorithemen auszuwählen
        und diese gegeneinander spielen zu lassen. Damit man dieses Spiel sehe kann, sollen alle Züge die 
        von beiden getätigt werden auf dem Spielbrett des Touch-Monitors angezeigt werden.
    \item \textbf{/F30/} \textit{Eine ELO Zahl soll die Spielstärke der Algorithmen angeben:} Damit der Benutzer eine für sich angemessene Herausforderung findet,
        sollen schwächere und stärkere KI-Algorithmen in der GUI gekennzeichnet werden. Um eine genaue Kennzahl für die Stärke zu erhalten, werden die ELO Zahlen durch
        Simulationen berechnet. Diese Simulationen sind Spiele der Algorithmen untereinander.
   
\end{itemize}
\subsubsection{Nichtfunktionale Anforderungen}
\begin{itemize}
    \item \textbf{/Q10/} \textit{Robustheit der Smarthphone Verbindung:} Nach der Verbindung mit dem Smarthphone (mittels QR-Code /F12/) muss sichergestellt sein, dass die Verbindung nicht ohne
        Grund abbricht, sondern erst, wenn z.B. die Distanz zwischen Smartphone und Pi zu groß ist. Des weitern soll nach einem Verbindungsabbruch, 
        das Spiel nicht abgebrochen werden, sondern es soll eine Möglichkeit zum Wiederverbinden bestehen.
    \item \textbf{/Q20/} \textit{ELO Zahlen sollen Stärke wiederspiegeln:} Die durch die Simulationen errechnete ELO Zahl von /F30/ soll auch in etwa dem Stärkegrad der Algorithmen entsprechen.
        Hat ein Algorithmus die sehr viel mehr ELO muss er auch dementsprechend stärker sein.
    \item \textbf{/Q30/} \textit{Zeiteinstellung soll von der KI Berücksichtigt werden:} Die Zeiteinstellung von /F13/ soll vom KI Client als Berechnungsdauer genutzt werden.
        Dieser soll dabei seine Rechenzeit so gut wie möglich an die Zeiteinstellung anpassen.
    \item \textbf{/Q40/} \textit{Reaktionszeit des Touch-Interfaces:} Das Berühren des Touch-Monitors soll zur sofortigen Ausführung des Befehls der Software führen.
    \item \textbf{/Q50/} \textit{Mobile Responsiveness:} Das Spielfeld soll auf egal welchem verbundenen Smarthphone gleich skaliert aussehen. 
        Das Verwenden von Tablets, oder das Drehen des Gerätes soll keinen Einfluss auf die Darstellung des Spielbrettes haben.
\end{itemize}
\subsubsection{Zusammenfassung der Anforderungen}
Die Identifizierten Funktionalen und Nichtfunktionalen Anforderungen werden in der Tabelle \ref{tab:Anforderungen} zusammengefasst. 
Die Kürzel sind für die folgenden Kapietel von Relevanz da sie in diesen Referenziert werden.
\vspace{1em}
\begin{table}[!h]
	\centering
	\begin{tabular}{|l|l|l|}
		\hline
		\textbf{ID} & \textbf{Funktionale Anforderung}\\
		\hline
		/F10/ & Menü zum Auswählen des Spieles \\
		\hline
		/F11/ & Auswählbares Menü für verschiedene KI Algorithmen \\
		\hline
        /F12/ & QR-Code fürs verbinden mit dem Smartphone \\
		\hline
		/F13/ & Varaible Zeiteinstellung im Menü \\
        \hline
		/F20/ & Benutzer soll Züge ausführen können \\
        \hline
		/F21/ & Benutzer sollen gegen andere Benutzer spielen können \\
        \hline
		/F22/ & Der Benutzer kann KI's gegeneinander spielen lassen \\
        \hline
		/F30/ & Eine ELO Zahl soll die Spielstärke der Algorithmen angeben \\
	
		\hline
		\textbf{ID} & \textbf{Nichtfunktionale Anforderung}\\
		\hline
		/Q10/ & Robustheit der Smarthphone Verbindung \\
        \hline
		/Q20/ & ELO Zahlen sollen Stärke wiederspiegeln \\
        \hline
		/Q30/ & Zeiteinstellung soll von der KI Berücksichtigt werden \\
        \hline
		/Q40/ & Reaktionszeit des Touch-Interfaces \\
        \hline
		/Q50/ & Mobile Responsiveness \\
		\hline
	\end{tabular}
	\caption{Anforderungstabelle}
	\label{tab:Anforderungen}
\end{table}

\section{UML Klassendiagramm vom Gesamten Projekt}
\label{apx:AllClassDiagrams}
\vspace{1em}
\begin{minipage}{\linewidth}
	\centering
	\includegraphics[width=1.0\linewidth]{pics/AllClassDiagrams.png}
	\captionof{figure}{ UML Klassendigramm aller Komponenten des Projekts }
\end{minipage}



\end{appendix}


\pagebreak




\end{document}
