% Kopf- und Fusszeile
\renewcommand{\sectionmark}[1]{\markright{#1}}
\renewcommand{\leftmark}{\rightmark}
\pagestyle{fancy}
\lhead{}
\chead{}
\rhead{\thesection\space\contentsname}
\lfoot{}
\cfoot{}
\rfoot{\ \linebreak Seite \thepage}
\renewcommand{\headrulewidth}{0.4pt}
\renewcommand{\footrulewidth}{0.4pt}

% Vorspann
\renewcommand{\thesection}{\Roman{section}}
\renewcommand{\theHsection}{\Roman{section}}
\pagenumbering{Roman}

\newcommand{\folgen}[1]{
\ensuremath
#1
}

\newcommandx{\student}[3][]{
	\def\studentName{#1}%
	\def\studentMatnr{#2}%
	\def\studentStudiengang{#3}%
}


\newcommandx{\MyTitelseite}[8][]{
\thispagestyle{empty}
\includegraphics[scale=0.2]{pics/oth-logo.png}\hfill
\IfFileExists{#1}{\includegraphics[scale=0.5]{#1}}{}
\begin{center}
\ifthenelse{\equal{#2}{2}}{ % then
	\vspace*{2cm}
	\Large
	\textbf{Ostbayerische Technische Hochschule Regensburg}\\
	\textbf{Fakultät für Informatik und Mathematik}\\
	\vspace*{2cm}
	\Huge
	\textbf{#3}\\[1em]
	\large
	Zur Erlangung des akademischen Grades des\\
	\ifthenelse{\equal{#3}{Bachelorarbeit}}{Bachelor of Science (B.Sc.)}{Master of Science (M.Sc.)}\\
	\vspace*{1cm}
	\Large
	\textbf{#4}\\
}{ % else
	\vspace*{1cm}
	\Large
	\textbf{#4}\\
	\vspace*{2cm}
	\large
	An der Fakultät für Informatik und Mathematik der\\
	Ostbayerischen Technischen Hochschule Regensburg\\
	im Studiengang\\
	\studentStudiengang\\[2em]
	eingereichte\\
	\vspace*{1cm}
	\Large
	\textbf{#3}\\[2em]
	\large
	zur Erlangung des akademischen Grades des\\
	\ifthenelse{\equal{#3}{Bachelorarbeit}}{Bachelor of Science (B.Sc.)}{Master of Science (M.Sc.)}
	\vspace*{1cm}
	\Large
}
	\vfill
	\normalsize
	%\newcolumntype{x}[1]{>{\raggedleft\arraybackslash\hspace{0pt}}p{#1}}
	\begin{tabular}{rl}%{6cm}p{7.5cm}}
	    \rule{0mm}{1ex}\textbf{Vorgelegt von:} & \studentName \\
		\rule{0mm}{1ex}\textbf{Matrikelnummer:} & \hspace*{-0.5em}\begin{tabular}[t]{r}\studentMatnr\end{tabular} \\ 
		\ifthenelse{\equal{#2}{1}}{~\\}{\rule{0mm}{1ex}\textbf{Studiengang:} & \studentStudiengang \\[2em]}
		\rule{0mm}{1ex}\textbf{Erstgutachter:} & #5 \\ 
		\rule{0mm}{1ex}\textbf{Zweitgutachter:} & #6 \\[2em]
		\rule{0mm}{1ex}\textbf{Abgabedatum:} & #7 \\ 
	\end{tabular} 
\end{center}
\pagebreak
}

\newsavebox\mybox
\savebox\mybox{%
  \tikz{
    \draw[ultra thick,red] (-4pt,-4pt) -- (4pt,4pt);
    \draw[ultra thick,red] (-4pt,4pt) -- (4pt,-4pt);
  }%
}  

\newcommand{\TopAlign}[1]{\adjustbox{valign=t}{#1}}
\newcolumntype{T}{>{\collectcell{\TopAlign}}c<{\endcollectcell}}



\forestset{
  upsideTria/.style={
    node format={
      \noexpand\node [
      draw,
      shape=regular polygon,
      regular polygon sides=3,
      inner sep=0pt,
      outer sep=0pt,
      \forestoption{node options},
      anchor=\forestoption{anchor}
      ]
      (\forestoption{name}) {\foresteoption{content format}};
    },
    child anchor=parent,
  },
  downsideTria/.style={
     node format={
      \noexpand\node [
      draw,
      shape=regular polygon,
      regular polygon sides=3,
      shape border rotate=180,
	  inner sep=0pt,
      outer sep=0pt,
      \forestoption{node options},
      anchor=\forestoption{anchor}
      ]
      (\forestoption{name}) {\foresteoption{content format}};
    },
    child anchor=parent, 
  },
  downsideTriaTerm/.style={
     node format={
      \noexpand\node [
      draw,
      shape=regular polygon,
      regular polygon sides=3,
      shape border rotate=180,
	  inner sep=0pt,
	  outer sep=0pt,
	  fill=green,
      \forestoption{node options},
      anchor=\forestoption{anchor}
      ]
      (\forestoption{name}) {\foresteoption{content format}};
    },
    child anchor=parent, 
  },
  upsideTriaTerm/.style={
    node format={
      \noexpand\node [
      draw,
      shape=regular polygon,
      regular polygon sides=3,
      inner sep=0pt,
      outer sep=0pt,
	  fill=green,
      \forestoption{node options},
      anchor=\forestoption{anchor}
      ]
      (\forestoption{name}) {\foresteoption{content format}};
    },
    child anchor=parent,
  },
  upsideTriaYellow/.style={
    node format={
      \noexpand\node [
      draw,
      shape=regular polygon,
      regular polygon sides=3,
      inner sep=0pt,
      outer sep=0pt,
	  fill=yellow,
      \forestoption{node options},
      anchor=\forestoption{anchor}
      ]
      (\forestoption{name}) {\foresteoption{content format}};
    },
    child anchor=parent,
  },
  downsideTriaYellow/.style={
     node format={
      \noexpand\node [
      draw,
      shape=regular polygon,
      regular polygon sides=3,
      shape border rotate=180,
	  inner sep=0pt,
	  outer sep=0pt,
	  fill=yellow,
      \forestoption{node options},
      anchor=\forestoption{anchor}
      ]
      (\forestoption{name}) {\foresteoption{content format}};
    },
	child anchor=parent,
  }
}

\tikzset{
myedge/.style={
  decoration={
   markings,
   mark=at position 0.5 with \node {\usebox\mybox};
  },
  postaction=decorate
  }
}